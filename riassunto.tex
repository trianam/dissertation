\documentclass{letter}
\usepackage[a4paper,left=1.5cm, right=1.5cm, top=5cm, bottom=2.5cm]{geometry}

\begin{document}
\pagenumbering{gobble}
\begin{tabular}{ll}
  \textbf{Cognome candidato: }& Martina\\
  \textbf{Nome candidato: }& Stefano\\
  &\\
  \textbf{Titolo della tesi: }& B-Spline methods for the design of smooth spatial
  paths with obstacle avoidance\\
  &\small(Metodi B-spline per il disegno di percorsi
  regolari in ambienti tridimensionali contenenti ostacoli)\\
  &\\
  \textbf{Relatore: }& Alessandra Sestini (alessandra.sestini@unifi.it)\\
  \textbf{Correlatore: }& Carlotta Giannelli
                          (carlotta.giannelli@unifi.it)\\
  &\\
  &\\
  &\\
  &\\
  \textbf{Riassunto: }&
\end{tabular}

\begin{center}
  \begin{minipage}{0.7\linewidth}
    Il problema del \emph{path planning} consiste nel trovare una curva
    che interpoli due punti dati, tale che non intersechi degli ostacoli in
    un ambiente. Tale problema ha significative applicazioni in robotica e
    visualizzazione scientifica. Inoltre, \`e importante avere
    certe qualit\`a di smoothness, quindi ci focalizziamo
    nell'ottenere una curva con delle buone caratteristiche di
    curvatura e torsione.

    Per la rappresentazione abbiamo usato le curve B-spline, che sono
    uno standard affermato nel mondo del Computer-Aided Design (CAD) e
    del Computer-Aided Geometric Design (CAGD).

    Abbiamo realizzato differenti algoritmi per risolvere il
    problema. Inoltre abbiamo elaborato la relativa analisi di
    complessit\`a. Il risultato \`e altamente interdisciplinare,
    abbiamo integrato, infatti, approcci analitici e stocastici.

    Abbiamo inoltre implementato tali algoritmi in una applicazione in
    Python 3, usando VTK come libreria grafica. Infine abbiamo testato
    sistematicamente, su differenti scenari, tale applicazione.
  \end{minipage}
\end{center}


\end{document}

%%% Local Variables:
%%% mode: latex
%%% TeX-master: t
%%% End:
