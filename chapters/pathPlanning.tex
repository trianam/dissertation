\documentclass[dissertation.tex]{subfiles}
\begin{document}
\chapter{Motion Planning}
The problem of \emph{motion planning} is to find a set of
low level tasks, given an high level goal to fulfill
\cite{choset}, for instance a classic motion planning problem is the
\emph{piano movers'} problem that involve moving in 3-dimensional space
a free flying rigid 
body from a start configuration to a goal configuration applying
translations and rotations and avoiding a set
of obstacles \cite{choset}\cite{lavalle}. The motion planning find
applications in 
different topics - robotics, \acp{UAV} \cite{goerzen}, autonomous
vehicles \cite{paden} are the most
famous applications - but it finds applications also in other less
common topics like motion of digital actors or drug design
\cite{choset}.

Initially the term motion planning referred only to the translations
and rotations of objects, ignoring the dynamics of them, but lately
the research on this field started considering also the physical
constraints of the object to move \cite{lavalle}. Usually the term
\emph{trajectory planning} refers to the problem of taking the
solution of a
motion planning algorithm and determine how to move a robot on it
respecting the mechanical constraints of the robot \cite{lavalle}.

For motion planning problems an important concept is the state space,
it can have different dimensions, one for each degree of freedom of
the robot, and it can be a discrete space or a continuos space (that
can be discretized) \cite{lavalle}. We can call the space state $\mS$,
and - considering that there are obstacles or constraints on the scene
- we can call $\mS_{free}\subseteq\mS$
the portion of the state space such that all the configuration in it
are free from obstacles or constraints. On this space state we have
the two special states $\ve{s}\in\mS$ and $\ve{e}\in\mS$ of the
desired \emph{start} and \emph{end} configurations.

Another concept is the geometric design of the scene and the
actor. The obstacles can be represented as convex
polygons/polyhedrons, complex polygons/polyhedron or also more complex
shapes \cite{lavalle}.

Also is important to define the transformation of the bodies, if is
possible only to translate and rotate the body or if it is composed of
rigid kinematic chains or trees or if is possible to have also not
rigid transformations (flexible materials) \cite{lavalle}.

\section{Problem types}
Many different problems are present in literature regarding motion
planning, we give on this section a short survey of the principal
problems ordered by increasing complexity. For details look
\cite{goerzen}.

\paragraph{Point vehicle}
On this problem the body of the object to move is represented as a
point in space, so the state space $\mS$ consists on the euclidean
space $\mS^2$ if we deal with land vehicles or $\mS^3$ if we deal with
aerial vehicles.

\paragraph{Point vehicle with differential constraints}
This problem extend the point vehicle's problem adding the
constraint of the physical dynamic, for instance constraint on
acceleration, velocity, curvature, etc\dots when we want to model a
real vehicle (that we continue to approximate with a point regarding
the shape).

\paragraph{Jogger's problem}
This kind of problems deal with dynamic of a jogger that have a
limited field of view, and so contrarily to the previous problems we
don't have a complete knowledge of the scene and the path is
updated in correspondence increase of knowledge of the scene.

\paragraph{Bug's problem}
This problem is an extreme case of the jogger's problem with a null
field of view, for updating the scene is necessary touching an
obstacle.

\paragraph{Weighted regions' problem}
On this problem we can have not completely obstructive obstacles, but
we have regions of the state space more desirable respect to
others. For instance this is the case of finding a path in an off-road
behavior where on some kind of terrains the vehicle go faster and on
other kind of terrain go slower.

\paragraph{Mover's problem}
Is the kind of problem of the cited before piano mover's problem. The
vehicle is modeled as a rigid body, and so we need to add to the state
space the
dimensions for the rotation of the body.

\paragraph{General vehicle with differential constraints}
Is a combination of the mover's problem and the point vehicle with
differential constraints, in the sense that we add to the mover's
problem also the physical constraints on the dynamic of the motion.

\paragraph{Time varying environments}
We need to deal with obstacles that move in time.

\paragraph{Multiple movers}
On this case we have more than one vehicle to move, so we need to deal
with different paths and to the problem of avoiding eventual
collisions between different path. In detail we admit collisions of
path on different times, but we need to avoid collisions on paths on
the same time.

\section{Algorithm types}
We can divide the algorithm's for motion planning in different types
regarding of which problem they resolve. Those algorithms are divided
in different categories.
For more details on the different algorithms look \cite{goerzen} and
\cite{choset}.

\subsection{Roadmap methods}
This kind of algorithms reduce the problem of motion planning to the
problem of search in graph. Basically they work fitting the state
space with some kind of graph using certain rules. However the
solution that they find is a polygonal chain.

\subsubsection{Visibility graph}
Build a graph were the nodes are the vertices of the obstacles, and
the edges connect two vertices if it don't cross any obstacle, then
find the shortest path on this graph. This
kind of algorithm find the optimal solution in plane but don't scale
properly in 3-dimensional space.

\subsubsection{Edge sample visibility graph}
This is a way of making an extension of visibility graph method to
3-dimensional space. Basically it works distributing a discrete set of points
along the edges of the obstacles, with a certain density, and then
calculating the visibility
graph and the shortest path on it. The drawback is that the solution
is not optimal.

\subsubsection{Voronoi roadmap}
It build a graph that is kept equidistant to the obstacles, using
\ac{VD} as base method for constructing it. We discuss \ac{VD}
in detail in \cref{sec:voronoi} and also we used a method that is
framed as a Voronoi roadmap that is described in \cref{sec:polChain}.

\subsubsection{Freeway method}
Also this method build a structure that is distant from the obstacles
fitting the free space with cylinders, then find the shortest path on
that structure.

\subsubsection{Silhouette method}
Developed by Canny, is not useful for practical uses but it is for
proving algorithmic bounds because it is proven to be complete in any
dimension. It work sweeping the space with a line (plane in
3-dimensional space) perpendicular to the
segment between $\ve{s}$ and $\ve{e}$ and building the shape of the
obstacles when the sweeping line intersect them.

\subsection{Cell decomposition}
Those methods decompose $\mS_{free}$ in smaller convex polygons - i.e.
trapezoids cylinders or balls - that
are connected by a graph, then search in such graph. It can be exact
or approximate, an exact cell decomposition method occupy all
$\mS_{free}$ with the graph structure, an approximate cell
decomposition method can occupy also portions of
$\mS\setminus\mS_{free}$ or all $\mS$, then the various polygons are
labelled as obstacle-empty, inside obstacle or partially occupied by
obstacles.

\subsection{Potential field methods}
This king of methods operate assigning a potential field on every
region of the space, the lowest potential is assigned to the goal
point $\ve{e}$ and to the obstacles is assigned an high potential
value. Then the path is calculated as a trajectory of a particle that
react to those potentials, it is repelled by the obstacles and
attracted by the end point.

\subsection{Probabilistic approaches}
Those kind of methods uses probabilistic techniques for exploring the
space of solutions and finding a good approximation of the optimal
solution. In our project we provided also a mixed
roadmap-probabilistic method, see \cref{sec:statisticalMethods} and
\cref{sec:inter3} for further details.

\subsection{\acf{RRT}}
This Method operates doing a stochastic search starting from the frame
of reference of the object to move and expanding a tree through the
random sampling of the state space.

\subsection{Decoupled trajectory planning}
This kind of algorithms operate in a two-step way, first a discrete
path through the state space is found, then the path is modified for
adapting it to the dynamics-constraints of the object to move.

\subsection{Mathematical programming}
Those methods threat the trajectory planning problem as a numerical
optimization problem, using methods like nonlinear programming for
finding the optimal solution.

\section{Path planning}
On our project we concentrated on a subsect of the motion planning
problem, the \emph{path planning} problem that consists \cite{choset}
of finding a parametric curve
\begin{equation*}
  \ve{C}(t)\ :\ [a,b]\subset\mR\ \rightarrow\ \mS
\end{equation*}
such that $\ve{C}(a)$ coincides with the desired starting
configuration, $\ve{C}(a)=\ve{s}$ the desired start configuration,
$\ve{C}(b)=\ve{e}$ the desired end configuration and the image of
$\ve{C}(t)$ is a
subset of $\mS_{free}$, in other words
\begin{equation*}
\ve{C}(t)\in\mS_{free}\quad \forall t\in[a,b].
\end{equation*}

In principle the space of the states $\mS$ can be of any dimension,
for instance if we focus on the piano movers' problem the state is
composed of 3 dimensions for the position and other 3 dimension for
the rotation of the object \cite{lavalle}. Also
the curve $\ve{C}$ can be parameterized in any way, the parameter
$t$ can represent for instance the time at which the object is in the
state $\ve{C}(t)$. In the specific we concentrated on the problem of
path planning where the state space is $\mS=\mE^3$ and the curve is
parameterized in $[0,1]$, so we find a curve from one point in the
euclidean 3-dimensional space $\ve{s}$ to another point $\ve{e}$
avoiding obstacles in the space - the object that we move is a point
without dimensions (and consequently without rotations).

\end{document}

%%% Local Variables:
%%% mode: latex
%%% TeX-master: "../dissertation"
%%% End:
