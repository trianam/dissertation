\documentclass[dissertation.tex]{subfiles}
\begin{document}
\chapter{Motion Planning}\index{Motion planning}
The problem of \emph{motion planning} consists in determining a set of
low level tasks, given an high level goal to be fulfilled
\cite{choset}. For instance, a classic motion planning problem is the
\emph{piano movers'} problem that involves the motion of a free flying rigid 
body in the 3-dimensional space
 from a start to a goal configuration by applying
translations and rotations and by avoiding collisions with a set
of obstacles \cite{choset}\cite{lavalle}. Motion planning finds
applications in 
different areas, like robotics, \acp{UAV} \cite{goerzen} and autonomous
vehicles \cite{paden}. These are the most
famous applications but it finds utilization also in other less
common areas like motion of digital actors or molecule design
\cite{choset}.

Initially, the term motion planning referred only to the translations
and rotations of objects, ignoring the dynamics of them, but lately
research in this field started considering also the physical
constraints of the object to be moved \cite{lavalle}. Usually, the term
\emph{trajectory planning} is used to refer to the problem of taking
the path produced by a
motion planning algorithm and determine the time law for moving a
robot on it by
respecting its mechanical constraints \cite{lavalle}.

An important concept for motion planning problems is the \emph{state
  space},
that can have different dimensions, one for each degree of freedom of
the object to move. It can be a discrete or a continuous space
\cite{lavalle}. We can call the space state $\mS$
and, considering that there are obstacles or constraints on the scene,
we can call $\mS_{free}\subseteq\mS$
the portion of the state space such that all its configurations
are admissible. On this space state we have
special states $\ve{s}\in\mS$ and $\ve{e}\in\mS$ for the
desired \emph{start} and \emph{end} configurations, respectively.

Another concept is the geometric design of the scene and the
actor. The obstacles can be represented as convex
polygons/polyhedrons, or also as more complex
shapes \cite{lavalle}.

Furthermore, it is important to define the possible admissible
transformations of the body, if it is 
possible only to translate and rotate it or if its motion is composed of
rigid kinematic chains or trees or if it is even possible to have not
rigid transformations (flexible materials) \cite{lavalle}.

\section{Problem types}\index{Motion planning!problem types}
Many different problems related to motion
planning have been introduced in literature. In this section we
present a short 
survey of the most relevant problems
in order of increasing complexity. Refer to \cite{goerzen} for
details.

\paragraph{Point vehicle}
The body of the object to be moved is represented as a
point in the space. Thus the state space $\mS$ consists in the euclidean
space $\mE^2$ if we consider land vehicles or $\mE^3$ if we consider
aerial vehicles.

\paragraph{Point vehicle with differential constraints}
This problem extends the point vehicle's problem by adding the
constraints of the physical dynamic. For instance, constraints on
acceleration, velocity, curvature, etc\dots when we want to model a
real vehicle (whose shape is still approximated with a point).

\paragraph{Jogger's problem}
This kind of problems concerns the dynamic of a jogger that has a
limited field of view. Consequently, in this case, we do not have a
complete view of the scene and the path is
updated as soon as the knowledge of the scene increases.

\paragraph{Bug's problem}
This problem is an extreme case of the jogger's problem with a null
field of view. Thus the scene updating can be done only when an
obstacle is touched.

\paragraph{Weighted regions' problem}
This problem considers some regions of the space as more desirable than
others, rather than contemplate completely obstructive obstacles. For
instance, this is the case of finding a path in an off-road 
environment where the vehicle can move faster on certain terrains and
slower over different configurations.

\paragraph{Mover's problem}
The vehicle is modeled as a rigid body, thus, we need to add the
dimensions for the spatial rotation of the body to the state
space.

\paragraph{General vehicle with differential constraints}
This problem combines the \emph{mover's problem} and the \emph{point
  vehicle with 
differential constraints} by adding to the mover's
problem the physical constraints on the motion dynamic.

\paragraph{Time varying environments}
These problems regards moving obstacles.

\paragraph{Multiple movers}
This problem considers more than one vehicle. We need
to manage
different paths and the problem of avoiding possible
collisions between different movers. As a matter of fact, we have to
avoid
collisions between the paths followed by different movers only if
the collision point is reached by the movers simultaneously.

\section{Algorithm types}\index{Motion planning!algorithm types}
We can divide the algorithms for motion planning in different types
taking into account the specific problem they resolve. The algorithms
belonging to a certain type can be further divided
in different categories.
For more details on the different algorithms see \cite{goerzen} and
\cite{choset}.

\subsection{Roadmap methods}
This kind of algorithms reduces the problem of motion planning to
graph search algorithms. The state space is approximated with a
certain graph in order to find a solution in terms of a polygonal
chain.

\subsubsection{Visibility graph}
The visibility graph is one of the most known roadmap methods. The
nodes of the graph correspond to the vertices of each polygonal
obstacles in the considered scenario. The edges of the graph
correspond to linear segments between pair of vertices that do not
intersect any obstacle. The Dijkstra's algorithm is then usually
considered to compute the \emph{shortest path} between two vertices of
the graph \cite{dijkstra}.  Note
that the shortest path associated to the visibility graph in a planar
configuration is the absolute shortest path from the start to the goal
position with respect to the considered scenario, see e.g.,
\cite{deberg}. While this method finds the optimal solution (with respect to a
\emph{distance} criterion) in the planar case, it does not properly
scale in a 3-dimensional setting.

\subsubsection{Edge sample visibility graph}
The edge sample visibility graph is an extension of the visibility
graph method to the 3-dimensional case. The main idea consists in
distributing a discrete set of points along the edges of the obstacles
by considering  a certain density. The visibility graph and the related
shortest path of this configuration are then computed, but the
corresponding solution is not as optimal as in the planar case.

\subsubsection{Voronoi roadmap}
This method builds a graph that is kept equidistant to the obstacles,
using
\acp{VD} as base method for constructing it. We discuss \acp{VD}
in detail in \cref{sec:voronoi} and Voronoi roadmap method in
\cref{sec:polChain}.

\subsubsection{Silhouette method}
This method was developed by Canny \cite{canny}. It is not useful for
practical uses but just for
proving algorithmic bounds because it is proven to be complete in any
dimension. It works sweeping the space with a line (plane in
3-dimensional space) perpendicular to the
segment between $\ve{s}$ and $\ve{e}$ and building the shape of the
obstacles when the sweeping line intersects them.

\subsection{Cell decomposition}
This method decomposes $\mS_{free}$ in smaller convex polygons - i.e.
trapezoids cylinders or balls - that
are connected by a graph, then searches a solution in such graph. A cell
decomposition method can be exact
or approximate, the former kind operates occupying all
$\mS_{free}$ with the graph structure, the latter one can occupy also
portions of
$\mS\setminus\mS_{free}$ or all $\mS$. Then the various polygons are
labelled as obstacle-empty, inside obstacle or partially occupied by
obstacles.

\subsection{Potential field methods}
This kind of methods operates assigning a potential field on every
region of the space, the lowest potential is assigned to the goal
point $\ve{e}$ and a high potential value is assigned to the
obstacles. Then the path is calculated as a trajectory of a particle that
reacts to those potentials, it is repelled by the obstacles and
attracted by the end point.

\subsection{Probabilistic approaches}
This kind of methods uses probabilistic techniques for exploring the
space of solutions and finding a good approximation of the optimal
solution. In our project we provide also a mixed
roadmap-probabilistic method, see \cref{sec:statisticalMethods} and
\cref{sec:inter3} for further details.

\subsection{\acf{RRT}}
This method operates by doing a stochastic search, starting from the
reference frame of the object to be moved and expanding a tree through the
random sampling of the state space.

\subsection{Decoupled trajectory planning}
This kind of algorithms operates in a two-step way. First a
discrete path through the state space is found, then the path is
modified to 
adapt it to the dynamics constraints - i.e. the trajectory is constructed.

\subsection{Mathematical programming}
This method manages the trajectory planning problem as a numerical
optimization problem, using methods like nonlinear programming to
find the optimal solution.

\section{Path planning}\index{Path planning}
In our project we concentrate on a subset of the motion planning
problem, the \emph{path planning} problem that consists \cite{choset}
in determining a parametric curve
\begin{equation*}
  \ve{C}\ :\ [a,b]\subset\mR\ \rightarrow\ \mS
\end{equation*}
such that $\ve{C}(a)=\ve{s}$ coincides with the desired starting
configuration,
$\ve{C}(b)=\ve{e}$ the desired end configuration and the image of
$\ve{C}$ is a
subset of $\mS_{free}$, in other words
\begin{equation*}
\ve{C}(u)\in\mS_{free}\quad \forall u\in[a,b].
\end{equation*}

In principle the space of the states $\mS$ can be of any dimension,
for instance if we focus on the piano movers' problem the state is
composed by 3 dimensions for the position and other 3 dimensions for
the rotation of the object \cite{lavalle}. Also
the curve $\ve{C}$ can be parameterized in any way. 

In this project we
concentrate on the problem of
path planning where the state space is $\mS=\mE^3$ and the curve is
parameterized in $[0,1]$. Thus we find a curve from one point
$\ve{s}\in\mE^3$ to another point $\ve{e}\in\mE^3$ 
avoiding obstacles. The object that we move is considered just as a
point.

\end{document}

%%% Local Variables:
%%% mode: latex
%%% TeX-master: "../dissertation"
%%% End:
