\documentclass[dissertation.tex]{subfiles}
\begin{document}
\chapter{Scene representation}\label{cha:scene}
The problem of how to represent the scene is basically to choose how
to represent the obstacles, the path, and how to memorize all the
elements.

\section{Basic elements and path}
First of all we designed the project with path planning in the space
in mind, so all the coordinates are in $\mE^3$. Also we concentrated
on \bss because we want that the output of the algorithm, the
path between a start point $s$ and an end point $e$, should be a \bs
curve. The elements for identify univocally a curve
$\ve{S}(t):\mR\rightarrow\mE^3$ are the
degree $m$ of the curve, the $n$ control vertices
\begin{equation*}
  P=\{\ve{v_0},\dots,\ve{v_n}\}
\end{equation*}
that are vectors in $\mE^3$, and finally the partition vector of the domain
\begin{equation*}
  T=\{t_0,\dots,t_{n-m+1}\}
\end{equation*}
with the knots $t_i$, where $t_m\equiv a$ and $t_{n+1}\equiv b$ are
the extremes of the domain $[a,b]$ of the curve.

Regarding the degree of the curve we gave
to the users the possibility of choosing between quadratic curves
($m=2$), cubic curves ($m=3$) and quartic curves ($m=4$). The users
choose also the starting point $s$ that coincides with $t_0$ and the
ending point $e$ that coincides with $t_n$. The number
of vertices and the other vertices themselves come from the algorithm,
and depends of the position of $s$ and $e$ and the obstacles, see
\cref{sec:polChain} for details.

The knots are generated automatically using one of two methods, who the
user choose, see \cref{sec:knotSel} for details.

So for the curve we memorize only the path $P$ and the degree
$m$. When we need to plot it we tabulate $\ve{S}(t)$ for a certain
number\footnote{Enough for having a smooth look.} of values of $t$ and
then we draw the polygonal chain that connect them.

\section{Basic obstacle representation}\index{Obstacle}\index{\acf{OTF}}
Apart the curve, in the scene we need to represent the obstacles, we
call $Obs$ the set of all obstacle in scene. We choose to represents
each obstacle $Ob\in Obs$ as a set of triangular faces that we call
\acp{OTF} each one containing three vertices. So recapping we have
\begin{equation*}
  \begin{split}
    Obs&=\{Ob_0,\dots,Ob_{\#Obs}\}\\
    Ob_i&=\{Otf_{i,0},\dots,Otf_{i,\#Otf_i}\}&i=0,\dots,\#Obs\\
    Otf_{i,j}&=\{\ve{p_{i,j,0}},\ve{p_{i,j,1}},\ve{p_{i,j,2}}\}\quad\qquad&
    i=0,\dots,\#Obs;\quad j=0,\dots,\#Otf_i
  \end{split}
\end{equation*}
where $\#Obs$ is the number of obstacles in the scene and $\#Otf_i$ is
the number of \acp{OTF} in obstacle $Ob_i$.

We choose this specific configuration because doing that the basic units that
constitutes the obstacles of the scene are triangles, and so all the
intersection that can occur are between triangle and triangle or
triangle and segment and so simple to calculate. If we wanted to
design the obstacles in a more complex way we should have provided
methods for triangulate the faces. 

The drawback of this approach is that if we want to insert
complex-shaped obstacles in the scene we need to triangulate the
surface in a manual way, and insert the final \acp{OTF}. However we
provided also methods
of abstracting this task that provide the possibility of creating more
complex obstacles in a simple way (\cref{sec:complexObs}).

Using this solution also means that we can potentially insert in the
scene open polyhedrons or shapes that intersect, as we don't have
any restriction on the position of the points $\ve{p_{i,j,k}}$.

\section{Tetrahedron and convex hull obstacles}\label{sec:complexObs}
For helping the user with the construction of the scene we realized
also two methods that builds obstacles in a simplified way
\begin{itemize}
\item one for building tetrahedrons;
\item and one more general for building convex hulls.
\end{itemize}

\begin{algo}{Abstract construction of tetrahedron}{alg:tetrahedron}
  \Procedure{buildTetrahedron}{$\ve{a}, \ve{b}, \ve{c}, \ve{d}$}
  \State $Obs\Ass Obs\cup\{\ \{\ve{a},\ve{b},\ve{c}\},\ \{\ve{a},\ve{b},\ve{d}\},\ \{\ve{b},\ve{c},\ve{d}\},\ \{\ve{c},\ve{a},\ve{d}\}\ \}$
  \EndProcedure
\end{algo}
\begin{algo}{Abstract construction of convex hull
    polyhedron}{alg:convHullPoly}
  \Procedure{buildConvexHullPolyhedron}{$\ve{p_0},\dots,\ve{p_n}$}
  \State $Ob\Ass\emptyset$
  \State $facets\Ass convexHull(\{\ve{p_0},\dots,\ve{p_n}\})$
  \ForAll{$f\in facets$}
  \State $simplices\Ass triangularize(f)$
  \ForAll{$s\in simplices$}
  \State $Ob\Ass Ob\cup\{\ve{s_0},\ve{s_1},\ve{s_2}\}$
  \EndFor
  \EndFor
  \State $Obs\Ass Obs\cup Ob$
  \EndProcedure
\end{algo}
\cref{alg:tetrahedron} take the four vertices of the tetrahedron to
build and add to $Obs$ a new obstacle that have all the faces of the
unique tetrahedron that can be built with the four
points.

\cref{alg:convHullPoly} is more complex, first we need to build the
convex hull of the input points (see \cite{deberg} and \cite{press}
for details on the 
convex hull algorithm), then we obtain a set of $facets$ that can be
triangles but also more complex planar polygons, in that case we need
to triangularize the $facets$ obtaining one or more $simplices$ that
are triangles (see \cite{deberg} and \cite{press} for details on the
triangularization
algorithms), finally we add each simplex as a new \ac{OTF} of the
obstacle.

\section{Bounding box}\index{Bounding box}
We decided also to give to the user the possibility of adding a
bounding box around the scene. It is built as an obstacle, using
\acp{OTF}, in fact we provide a method that take two points $\ve{a}$
and $\ve{b}$ and build the cube with extremes those points and with
all the faces triangularized like in \cref{fig:boundingBox}.
\begin{myfig}{Bounding box with extremes $\ve{a}$ and $\ve{b}$.}{fig:boundingBox}
  \begin{tikzpicture}
    \pgfmathsetmacro\front{6}
    \pgfmathsetmacro\side{2}
    \pgfmathsetmacro\sum{\front+\side}

    \coordinate (a) at (0,0);
    \coordinate (o1) at (\front,0);
    \coordinate (o2) at (\sum,\side);
    \coordinate (o3) at (\front,\front);
    \coordinate (o4) at (0,\front);
    \coordinate (o5) at (\side,\sum);
    \coordinate (b) at (\sum,\sum);


    \draw[poly] (a) -- (o1) -- (o3) -- (o4) -- (a);
    \draw[poly] (o1) -- (o2) -- (b) -- (o3);
    \draw[poly] (b) -- (o5) -- (o4);
    \draw[poly] (o4) -- (o1);
    \draw[poly] (b) -- (o1);
    \draw[poly] (b) -- (o4);


    \foreach \p in {o1,o2,o3,o4,o5}
    \filldraw[vertex] (\p) circle (2pt);

    \foreach \p in {a,b}
    \filldraw[vertexHigh] (\p) circle (4pt);

    \foreach \n/\l/\p in {a/a/{below left},b/b/{above
        right},o1/n_1/below,o2/n_2/{below right},o3/n_3/{below
        right},o4/n_4/{left},o5/n_5/{above left}}
    \node[\p] at (\n) {$\ve{\l}$};
  \end{tikzpicture}
\end{myfig}

In all the project, regarding the intersections, the \acp{OTF} of the
bounding box are considered
exactly like the \acp{OTF} of the obstacles, the only differences are
that the bounding box is not visible when the scene is plotted, and a
point inside the bounding box is not considered inside obstacle.

\end{document}


%%% Local Variables:
%%% mode: latex
%%% TeX-master: "../dissertation"
%%% End:

