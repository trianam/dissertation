\documentclass[dissertation.tex]{subfiles}
\begin{document}
\chapter{Scene representation}\label{cha:scene}
The problem of scene description basically consists in fixing a
representation of the obstacles and of the path, besides in establishing the structures adopted for their storage.

\section{Basic elements and path}
First of all, since we are interested in spatial path planning, all the point coordinates are in $\mE^3$. Also we concentrate
on \bss because we want a standard representation for the output of
the algorithm, the path between a start point $s$ and an end point
$e$. Actually \bss
curves are the standard adopted in \ac{CAD} and
\ac{CAGD} systems \cite{hughes}\cite{foley}. 

Now the structures which  uniquely identify a \bs curve are
three: its degree m, the associated control polygon and the extended
knot vector 

Regarding the degree of the curve, we give
to the users the possibility of choosing among quadratics
($m=2$), cubics ($m=3$) and quartics ($m=4$). The users
choose also the starting and ending points $s$ and $e$ respectively,
associated with the parameter values $t_0=\dots=t_m$ and
$t_{n+1}=\dots=t_{n+m+1}$.

The number
of vertices and the other vertices themselves come from the algorithm,
and they depend on the position of $s$ and $e$ and on the obstacles, see
\cref{sec:polChain} for details.

The knots are generated automatically using one of two methods
described in \cref{sec:knotSel}.

So for the curve we memorize only the control vertices $P$ and the
degree 
$m$. As usual in any computer graphic system, when we want its plotting, we tabulate $\ve{S}$ for a certain
number\footnote{Enough for having a smooth look.} of values of $t$ and
then we draw the polygonal chain that connects them.

\section{Basic obstacle representation}\index{Obstacle}\index{\acf{OTF}}
Apart the curve, in the scene we need to represent the obstacles. We
call $Obs$ the set of all obstacle in scene. We choose to represent
each obstacle $Ob\in Obs$ as a set of triangular faces that we call
\acp{OTF}, each one containing three vertices. Resuming we have
\begin{equation*}
  \begin{split}
    Obs&=\{Ob_0,\dots,Ob_{\#Obs}\}\\
    Ob_i&=\{Otf_{i,0},\dots,Otf_{i,\#Otf_i}\}&i=0,\dots,\#Obs\\
    Otf_{i,j}&=\{\ve{p_{i,j,0}},\ve{p_{i,j,1}},\ve{p_{i,j,2}}\}\quad\qquad&
    i=0,\dots,\#Obs;\quad j=0,\dots,\#Otf_i
  \end{split}
\end{equation*}
where $\#Obs$ is the number of obstacles in the scene and $\#Otf_i$ is
the number of \acp{OTF} in obstacle $Ob_i$.

We choose this specific configuration because in this way all the
intersections that can occur are between triangle and triangle or
triangle and segment and they can be easily calculated. This implies
that, if an obstacle is a polyhedron more complex than just a
tetrahedron, preliminarily its faces have to be triangulated.

We provide the methods explained in \cref{sec:complexObs} for
abstracting the creation of \acp{OTF}.

Using this solution also means that we can potentially insert in the
scene open polyhedrons\footnote{For instance a tetrahedron without one
  face} or intersecting shapes, as we don't have
any restriction on the position of the points $\ve{p_{i,j,k}}$.

\section{Complex obstacles}\label{sec:complexObs}
In order to simplify the scene construction, four methods for easily
building obstacles have been realized: 
\begin{itemize}
\item one for building tetrahedrons;
\item one for building parallelepipeds\footnote{aligned with the axis};  
\item one more general for building convex hulls;
\item a special method used for building a bucket-shaped obstacle that
  we use in the tests.
\end{itemize}

\begin{algo}{Abstract construction of tetrahedron}{alg:tetrahedron}
  \Procedure{buildTetrahedron}{$Obs, \ve{a}, \ve{b}, \ve{c}, \ve{d}$}
  \State $Obs\Ass Obs\cup\{\ \{\ve{a},\ve{b},\ve{c}\},\ \{\ve{a},\ve{b},\ve{d}\},\ \{\ve{b},\ve{c},\ve{d}\},\ \{\ve{c},\ve{a},\ve{d}\}\ \}$
  \EndProcedure
\end{algo}
\begin{algo}{Abstract construction of convex hull
    polyhedron}{alg:convHullPoly}
  \Procedure{buildConvexHullPolyhedron}{$Obs, \ve{p_0},\dots,\ve{p_n}$}
  \State $Ob\Ass\emptyset$
  \State $facets\Ass convexHull(\{\ve{p_0},\dots,\ve{p_n}\})$
  \ForAll{$f\in facets$}
  \State $simplices\Ass triangularize(f)$
  \ForAll{$\{\ve{s_0},\ve{s_1},\ve{s_2}\}\in simplices$}
  \State $Ob\Ass Ob\cup\{\ve{s_0},\ve{s_1},\ve{s_2}\}$
  \EndFor
  \EndFor
  \State $Obs\Ass Obs\cup Ob$
  \EndProcedure
\end{algo}
\cref{alg:tetrahedron} takes the four vertices of a tetrahedron to
and adds to $Obs$ a new obstacle that have all the faces of the
unique tetrahedron that can be built with the four
points.

\cref{alg:convHullPoly} is more complex, first we need to build the
convex hull of the input points (see \cite{deberg} and \cite{press}
for details on the 
convex hull algorithm), then we obtain a set of $facets$ that have to
be triangulated (see \cite{deberg} and \cite{press} for details on the
triangularization
algorithms). Finally we add each triangle as a new \ac{OTF} of the
obstacle.

\section{Bounding box}\index{Bounding box}
We decided also to give to the user the possibility of adding a
bounding box around the scene. It is built as an obstacle, using
\acp{OTF}, in fact we provide a method that take two points $\ve{a}$
and $\ve{b}$ and build the parallelpiped with extremes those points and with
all the faces triangularized like in \cref{fig:boundingBox}.
\begin{myfig}{Bounding box with extremes $\ve{a}$ and $\ve{b}$.}{fig:boundingBox}
  \begin{tikzpicture}
    \pgfmathsetmacro\front{6}
    \pgfmathsetmacro\side{2}
    \pgfmathsetmacro\sum{\front+\side}

    \coordinate (a) at (0,0);
    \coordinate (o1) at (\front,0);
    \coordinate (o2) at (\sum,\side);
    \coordinate (o3) at (\front,\front);
    \coordinate (o4) at (0,\front);
    \coordinate (o5) at (\side,\sum);
    \coordinate (b) at (\sum,\sum);


    \draw[poly] (a) -- (o1) -- (o3) -- (o4) -- (a);
    \draw[poly] (o1) -- (o2) -- (b) -- (o3);
    \draw[poly] (b) -- (o5) -- (o4);
    \draw[poly] (o4) -- (o1);
    \draw[poly] (b) -- (o1);
    \draw[poly] (b) -- (o4);


    \foreach \p in {o1,o2,o3,o4,o5}
    \filldraw[vertex] (\p) circle (2pt);

    \foreach \p in {a,b}
    \filldraw[vertexHigh] (\p) circle (4pt);

    \foreach \n/\l/\p in {a/a/{below left},b/b/{above
        right},o1/n_1/below,o2/n_2/{below right},o3/n_3/{below
        right},o4/n_4/{left},o5/n_5/{above left}}
    \node[\p] at (\n) {$\ve{\l}$};
  \end{tikzpicture}
\end{myfig}

In all the project, regarding the intersections, the \acp{OTF} of the
bounding box are considered
exactly like the \acp{OTF} of the obstacles. The only differences are
that the bounding box is not visible when the scene is plotted, and a
point inside the bounding box is not considered inside obstacle.

\end{document}


%%% Local Variables:
%%% mode: latex
%%% TeX-master: "../dissertation"
%%% End:

