\documentclass[dissertation.tex]{subfiles}
\begin{document}
\chapter{Prerequisites}
\section{Splines and \bss}\label{sec:spline}
A \emph{spline} is a piecewise polynomial function with prescribed
regularity on its domain.

More formally we define a spline \cite{deboor}\cite{farin}\cite{salomon}\cite{bartels}
$$s:[a,b]\subset\mR\rightarrow\mR$$
as follows.
We have a partition of that interval defined by the \emph{breakpoints}
$$\tau = \{\tau_0,\dots,\tau_l\}$$
such that $a=\tau_0<\tau_1<\dots<\tau_{l-1}<\tau_l=b$ forming $l$
intervals
$$
I_i=
\begin{cases}
  [\tau_i,\tau_{i+1}) & \mbox{if } i=0,\dots,l-2\\
    [\tau_i,\tau_{i+1}] & \mbox{if } i=l-1\\
\end{cases}
$$
is possible to define the following spaces:
\paragraph{Piecewise polynomial functions space} $P_{m,\tau}$
is the space of the functions that are polynomials of maximum degree $m$
in each interval $I_i$ of the partition, formally:
\begin{multline*}
  P_{m,\tau}=\{f:[a,b]\rightarrow\mR\ \mid\ \exists p_0\dots
  p_{l-1}\in\Pi_m \ \text{such that}\\
  f(t)=p(t),\ \forall t\in I_i,\
  i=0\dots l-1\}
\end{multline*}
where $\Pi_m$ is the space of the polynomials of degree $\le m$. The
dimension of $P_{m,\tau}$ is
\begin{equation*}
  dim(P_{m,\tau})=l(m+1)
\end{equation*}
because the dimension of $\Pi_m$
is $m+1$.
\paragraph{Classic spline space}\index{Classic splines} $S_{m,\tau}$ is the space of
the piecewise polynomial functions of degree $m$ that have continuity
$C^{m-1}$ in
the junctions of the intervals, formally:
$$
S_{m,\tau}=P_{m,\tau}\cap C^{m-1}[a,b].
$$
The dimension of this space is
\begin{equation}
  \label{eq:dimClassicSpline}
  l(m+1)-(l-1)\cdot m\,=\,l+m.
\end{equation}

\paragraph{Generalized spline space}\index{Generalized splines} $S_{m,\tau,M}$ is the
space of piecewise polynomial function of degree $m$ with a prescribed
regularity at each breakpoint ranging from $-1$ to $m-1$. The
regularity is prescribed by the multiplicity vector
$$
M=\{m_1,\dots,m_{l-1}\},\quad m_i\in\mN,\quad 1\leq m_i\leq m+1
$$
as follows,
\begin{multline*}
  S_{m,\tau,M}=\{f:[a,b]\rightarrow\mR\ \mid\ \exists p_0\dots
  p_{l-1}\in\Pi_m \ \text{such that}\\
  f(t)=p(t),\ \forall t\in I_i,\
  i=0\dots l-1\ \text{and}\\
  p_{i-1}^{(j)}(\tau_i)=p_{i}^{(j)}(\tau_i),\ j=0,\dots,m-m_i,\ i=1,\dots,l-1\}.
\end{multline*}
The dimension of the space is equal to 
\begin{equation*}
  dim(S_{m,\tau,M})=l(m+1)-\sum_{i=1}^{l-1}(m-m_i+1)=m+\mu +1\qquad(\mu=\sum_{i=1}^{l-1}m_i)  
\end{equation*}

and is true that
$$
\Pi_m\subseteq S_{m,\tau}\subseteq S_{m,\tau,M}\subseteq P_{m,\tau},
$$
in particular:
\begin{itemize}
  \item if $m_i=1$ for all $i=1,\dots,l-1$, then
    $S_{m,\tau,M}=S_{m,\tau}$;
  \item if $m_i=m+1$ for all $i=1,\dots,l-1$, then
    $S_{m,\tau,M}=P_{m,\tau}$.
\end{itemize}

\subsection{Truncated-powers basis for classic
  splines}\label{sec:truncpow}\index{Classic splines!truncated-powers basis}
A truncated power $(t-\tau_i)_+^m$ is
defined by
$$
(t-\tau_i)_+^m=
\begin{cases}
  0,&\mbox{if}\quad t\leq\tau_i\\
  (t-\tau_i)^m, &\mbox{otherwise}.
\end{cases}
$$
Is possible to demonstrate that the functions
$$
g_i(t)=(t-\tau_i)_+^m)\ \in S_{m,\tau},\quad i=1,\dots,l-1
$$
are linearly independents, and that
$$
1,t,t^2,\dots,t^m,(t-\tau_1)_+^m,\dots,(t-\tau_{l-1})_+^m
$$
form a basis for the classic spline space \cite{deboor}. Then a
generic element $s\in S_{m,\tau}$ can be expressed as follows,
\begin{equation}\label{eq:classicSplineElement}
  s(t)=\sum_{i=0}^m c_i t^i\, +\, \sum_{j=1}^{l-1} d_j
  (t-\tau_j)_+^m\qquad
  \begin{split}
    &c_i\in\mR,\ i=0,\dots,m\\
    &d_j\in\mR,\ j=1,\dots,l-1.
  \end{split}
\end{equation}

\subsection{\bss basis for classic
  splines}\label{sec:bsplines}\index{\bss}\index{Classic splines!\bss basis}
\emph{\bss} are a specific basis wich can be alternatively used to
represent any generalized spline \cite{deboor}\cite{farin}\cite{salomon}\cite{bartels}. In this
paragraph however we consider only their definition to generate the
classic spline space $S_{m,\tau}$. Furthermore in some textbook, for
notational convenience, is considered the
\emph{order}$=m+1$.

For defining the \bss \cite{deboor} we need to extend the partition vector
$\tau=\{\tau_0,\cdots,\tau_l\}$ with $m$ knots to the left and $m$ to
the right, so we define a new vector, usually called \emph{extended knot} vector.
$$
T=\{t_0,\dots,t_{m-1},t_{m},\dots,t_{n+1},t_{n+2},\dots,t_{n+m+1}\}
$$
such that
\begin{equation*}
  t_0\leq\dots\leq t_{m-1}\leq \overmath{\equiv\tau_0\equiv a}{t_{m}}<\dots<\overmath{\equiv\tau_l\equiv b}{t_{n+1}}\leq t_{n+2}\leq\dots\leq t_{n+m+1}.    
\end{equation*}
$\tau$ have $l+1$ elements, so we can calculate the value of
$$
n=l+m-1,
$$
thus the dimension of $S_{m,\tau}$, for
\cref{eq:dimClassicSpline}, is 
\begin{equation*}
  dim(S_{m,\tau})=l+m=n+1  
\end{equation*}

The $n+1$ basis $N_{i,m+1}(t)$ of the \bss of degree $m$ are defined,
for $i=0,\dots,n$, by the recursive formula:
\begin{align*}
  N_{i,1}(t) &=
  \begin{cases}
    1,\quad \mbox{if}\quad t_i\leq t<t_{i+1}\\
    0,\quad \mbox{otherwise}
  \end{cases}\\
  N_{i,r}(t) &= \omega_{i,r-1}(t)\cdot N_{i,r-1}(t)\ +\
  (1-\omega_{i+1,r-1}(t))\cdot N_{i+1,r-1}(t)\\
             &\pushright r=2,\dots,m+1
\end{align*}
where
$$
\omega_{i,r}(t) = \begin{cases}
  \frac{t-t_i}{t_{i+r}-t_i},&\mbox{if }t_i\neq t_{i+r}\\
  0, &\mbox{otherwise.}
\end{cases}
$$

Then any function $s\in S_{m,\tau}$ can be expressed also as a linear
combination of \bss,
\begin{equation}\label{eq:bsplineElement}
  s(t)=\sum_{i=0}^nv_i N_{i,m+1}(t)\qquad, v_i\in\mR, i=0,\dots,n.
\end{equation}

\subsection{Spline curves}\index{Spline curves}
A \emph{spline curve} in the affine space $\mE^d$ is the image of a
parametric vector function $\ve{S}:[a,b]\rightarrow\mE^d$ whose
components are all splines belonging to a fixed spline space
$S_{m,\tau}$. For $d=3$
\begin{equation}\label{eq:bsplineCurveComp}
  \ve{S}(u) = \left[
    \begin{array}{c}
      \ve{x}(u)\\
      \ve{y}(u)\\
      \ve{z}(u)
    \end{array}
    \right].
\end{equation}

$\ve{S}(t)$ can be written as follows in the
truncated-powers basis
% A parametric spline \emph{curve} $\ve{S}(t)$ is a curve in a certain
% dimension obtained applying a spline function to a set of points
% in the same dimension
% called \emph{control vertices}. We can apply the control vertices to a
% spline expressed with the truncated powers as in
\cref{eq:classicSplineElement} replacing the coefficients $c_i$ and
$d_i$ with points
\begin{equation}\label{eq:classicSplineCurve}
  \begin{multlined}
  \ve{S}(t)=\sum_{i=0}^m \ve{c_i}\cdot t^i\, +\, \sum_{j=1}^{l-1}
  \ve{d_j}\cdot (t-\tau_j)_+^m\qquad,\ve{c_i}\in\mE^d,\
  \ve{d_j}\in\mE^d\\
  i=0,\dots,m;\ j=0,\dots,l-1
  \end{multlined}
\end{equation}
where this representation is not practical because
there isn't an intuitive correlation between the points
$\ve{c_i}$, $\ve{d_j}$ and the curve itself. Moreover the
determination of an interpolant to asset of argued points in $\mE^d$
is not a well conditioned problem if this form is adopted
\cite{deboor}. To overcome those drawbacks we can use \emph{\bss
  basis} (\cref{sec:bsplines}).

We can apply control vertices to a spline expressed with the \bs
basis as in \cref{eq:bsplineElement} replacing the coefficients $v_i$
with points, in this case $\ve{S}(t)$ is represented as follows
\begin{equation}\label{eq:bsplineCurve}
  \ve{S}(t)=\sum_{i=0}^n\ve{v_i}\cdot N_{i,m+1}(t)\qquad
  ,\ve{v_i}\in\mE^d,\ i=0,\dots,n.
\end{equation}
The representation of \cref{eq:bsplineCurve} is more convenient
than the previous one (\cref{eq:classicSplineCurve}) because the curve
$\ve{S}(t)$ roughly 
follow the shape given by the points $\ve{v_i}$. Those points are
called \emph{control vertices} and the polygon defined by them is
called \emph{control polygon} and they can be used to control the
shape of the curve.

\section{\bss curves properties}\label{sec:bsplineProp}\index{\bss!properties}
In this section we describe some properties of \bs curves that we
used for the development of the project.

\subsection{Convex hull}\label{sec:convexHull}\index{\bss!convex hull}
A \bs curve $\ve{S}(t)$ of order $m$ defined by the control polygon
$\ve{v_0},\ve{v_1},\dots,\ve{v_n}$ is contained inside the union of the
convex hulls composed of $m+1$ vertices of the control polygon
\cite{farin}. If we
call $\conv(\ve{w_0},\ve{w_1},\dots,\ve{w_j})$ the convex hull of the
vertices $\ve{w_0},\ve{w_1},\dots,\ve{w_j}$ then we have
\begin{eqnarray*}
  C_0&=&\conv(\ve{v_0},\ve{v_1},\dots,\ve{v_{m}})\\
  C_1&=&\conv(\ve{v_1},\ve{v_2},\dots,\ve{v_{m+1}})\\
  &\cdots&\\
  C_{n-m}&=&\conv(\ve{v_{n-m}},\ve{v_{n-m+1}},\dots,\ve{v_{n}})\\
\end{eqnarray*}
and the area where $\ve{S}(t)$ is contained is
\begin{equation*}
  C=C_0\cup C_1\cup C_{n-m}
\end{equation*}
or in other words must be true
\begin{equation*}
  \ve{S}(t)\cap C =\ve{S}(t)\qquad \forall t\in[a,b]
\end{equation*}
whatever is the partition vector.

\begin{myfig}{Convex hull containing \bs of grade 2}{fig:convexHull}
  \begin{tikzpicture}
    \coordinate (a) at (0,0);
    \coordinate (b) at (0.5,1.7);
    \coordinate (c) at (2,3);
    \coordinate (d) at (4,3);
    \coordinate (e) at (5.5,1.7);
    \coordinate (f) at (6,0);
    
    \path[convexHull] (a) -- (b) -- (c) -- (a);
    \path[convexHull] (b) -- (c) -- (d) -- (b);
    \path[convexHull] (c) -- (d) -- (e) -- (c);
    \path[convexHull] (d) -- (e) -- (f) -- (d);
    
    \draw[convexHullBord] (a) -- (c);
    \draw[convexHullBord] (b) -- (d);
    \draw[convexHullBord] (c) -- (e);
    \draw[convexHullBord] (d) -- (f);
    
    \draw[controlPoly] (a) -- (b) -- (c) -- (d) -- (e) -- (f);
    \foreach \p in {a,b,c,d,e,f}
    \filldraw[controlVert] (\p) circle (2pt);

    \node[above left] at (a) {$\ve{v_0}$};
    \node[above left] at (b) {$\ve{v_1}$};
    \node[above] at (c) {$\ve{v_2}$};
    \node[above] at (d) {$\ve{v_3}$};
    \node[above right] at (e) {$\ve{v_4}$};
    \node[above right] at (f) {$\ve{v_5}$};
  \end{tikzpicture}  
\end{myfig}
On \cref{fig:convexHull} is visible an example of a control polygon
and the surface where a quadratic \bs applied to
it can be contained. Note that the convex hull property holds also in
3-dimensional
space - i.e. a quadratic \bs in 3-dimensional space is contained
inside a flat surface composed by the union of triangles. From grade 3
the area where $\ve{S}(t)$ can be contained is not anymore plane
because is composed of union of solid polyhedrons.

\subsection{Smoothness}\label{sec:smoothness}\index{Smoothness}\index{\bss!smoothness}
The concept of smoothness consist of identifying what is the number of
possible derivatives of the function such that the function is
continue. A function $f$ that is not continue is said to be of class
$C^{-1}$, a function that is continue until derivative $d$ is said to
be of class $C^d$, a function that is always continue for every
derivative is said to be of class $C^\infty$.

A \bs curve of degree $m$ with $n$ control vertices consists of
$n-m$ polynomial segments, one for each 
interval 
\begin{equation*}
[t_i,t_{i+1}]\qquad i=m,\dots,n+1  
\end{equation*}
this mean that $\ve{S}(t)$ is $C^\infty$ for
\begin{equation*}
t\in(t_i,t_{i+1})\qquad i=m,\dots,n+1.
\end{equation*}
Note that, if we use generalized \bs curves, an interval
$[t_i,t_{i+1}]$ can also be of dimension $1$ if
we have a knot multiplicity $>1$, in
such case there isn't a polynomial segment.
On every breakpoint $t_i$ with $i=1,\dots,n+m-1$ we have that the curve has
smoothness $C^{m-1}$
and, if we use generalized \bs curves, smoothness $C^{m-r}$ where $r$ is the
multiplicity of the knot \cite{farin}.

In our project we don't use generalized \bs curves, so globally a
curve of degree $m$ have smoothness
\begin{equation*}
  C^{m-1}.
\end{equation*}

\subsection{Aligned vertices}\index{\bss!aligned vertices}\label{sec:alignedVertices}
In our project we dealt sometimes with some problems whose solution
involved aligning two or more control vertices of the curve. In this
section we analyze the effects on the curve of aligning vertices.

We can have the following situations:
\paragraph{$m$ aligned control vertices}
If $m$ control vertices $v_i,\dots,v_{i+m-1}$ of control polygon are on the same line then
the curve $\ve{S}(t)$ touch the segment joining those vertices.

\paragraph{$m+1$ aligned control vertices}
If $m+1$ control vertices $v_i,\dots,v_{i+m}$ of control polygon are on the same line then
the curve $\ve{S}(t)$ lay on the segment joining those vertices.

\subsection{End point interpolation}\index{\bss!end point interpolation}\label{sec:clamped}
In general a \bs curve with control vertices
\begin{equation*}
  \ve{v_0}, \dots, \ve{v_n}
\end{equation*}
and extended knot vector
\begin{equation*}
  T=\{t_0,\dots,t_{m-1},t_{m},\dots,t_{n+1},t_{n+2},\dots,t_{n+m+1}\}
\end{equation*}
does not necessarily interpolate any control vertex $v_i$, neither the
first and the last one. But we are
interested in using \bs for representing paths from one point to
another. So it should be a nice feature to have that the curve defined in
the domain $[a,b]$ is shaped such that
\begin{equation}\label{eq:clamped}
  \begin{cases}
    \ve{S}(t)=\ve{v_0}& \text{for } t=a\\
    \ve{S}(t)=\ve{v_n}& \text{for } t=b.
  \end{cases}
\end{equation}

We can obtain \cite{deboor} the conditions of \cref{eq:clamped} if we impose on the
extended partition vector $T$:
\begin{equation*}
  t_0=\dots= t_{m-1}= \overmath{\equiv a}{t_{m}}<\dots<
\overmath{\equiv b}{t_{n+1}}= t_{n+2}=\dots= t_{n+m+1}
\end{equation*}
in other words we want
\begin{equation*}
  T=\{\overbrace{a,\dots,a}^m,t_{m+1},\dots,t_{n},\overbrace{b,\dots,b}^m\}
\end{equation*}

\subsection{Curvature and
  torsion}\index{Curvature and torsion}\index{\bss!curvature and torsion}
Since we are interested in comparing different curves,
we need to recognize if a certain curve is a \emph{good} or a
\emph{bad} one. One factor that characterizes a certain curve can 
be its smoothness (\cref{sec:smoothness}) - i.e. a $C^3$ curve
is better than a $C^2$ curve - but this isn't
enough for comparing curves. Usually are used for this
\emph{curvature} and 
\emph{torsion} \cite{docarmo}\cite{salomon}. Both are scalar quantities defined on
sufficiently smooth 
parametric curves for each value of the parameter, and they do not
depend on the selected parametrization.

For a generic parametric curve $\ve{S}(u)$ defined for $u\in[a,b]$
given the notation $\wedge$ for the vector product and for \cref{eq:bsplineCurveComp}
\begin{equation*}
  \dot{\ve{S}}(u) = \frac{\md}{\md u} \ve{S}(u)=\left[
    \begin{array}{c}
      \frac{\md}{\md u} \ve{x}(u)\\
      \frac{\md}{\md u} \ve{y}(u)\\
      \frac{\md}{\md u} \ve{z}(u)
    \end{array}
    \right].
\end{equation*}
we define the
curvature $\kappa(u)$ and, in points with non vanishing curvature,
the torsion $\tau(u)$ as
\begin{empheq}[left={=\empheqbiglbrace~}]{align}
  \kappa(u) &= \frac{\norm{\dot{\ve{S}}(u)\wedge\ddot{\ve{S}}(u)}}{{\norm{\dot{\ve{S}}(u)}}^3}  \label{eq:curvature}\\
  \tau(u) &= \frac{\det\left[\dot{\ve{S}}(u),\ddot{\ve{S}}(u),\dddot{\ve{S}}(u)\right]}{\norm{\dot{\ve{S}}(u)\wedge\ddot{\ve{S}}(u)}} = \frac{\left(\dot{\ve{S}}(u)\wedge\ddot{\ve{S}}(u)\right)\cdot\dddot{\ve{S}}(u)}{\norm{\dot{\ve{S}}(u)\wedge\ddot{\ve{S}}(u)}}  \label{eq:torsion}
\end{empheq}

\cref{eq:curvature} and \cref{eq:torsion} describe completely the
behavior of $\ve{S}(u)$ locally for each value of $u$. Curvature and
torsion have also a geometric interpretation: for each value $\tilde{u}$ of
the parameter $u$, the inverse $\frac{1}{\kappa(\tilde{u})}$ of the
curvature is the radius of curvature on the point defined by
$\ve{S}(\tilde{u})$ - i.e. the radius of the 
osculating circle tangent in that point to the curve belongs to the
plane where the
curve is bending and
that is situated on the inner side of its turn - $\tau(\tilde{u})$ indicate
(if $\kappa(\tilde{u})\neq 0$) how sharply the plane where
the curve lies is rotating.

The value of $\kappa(u)$ can be only non negative, while $\tau(u)$ is
a signed quantity.

Two curves of same smoothness can be compared using the plots of
curvature and torsion, in general curves that have lower peaks of
$\kappa(u)$ and $\tau(u)$ are better than curves with
higher peaks.

\section{Voronoi Diagrams}\label{sec:voronoi}\index{\acfp{VD}}
In this section we introduce \acfp{VD}, an important structure used in
the project. \acp{VD} \cite{deberg} provide a method for creating a
partition of the
space using distances from a set of input points called
\emph{sites}. Formally we have a set
\begin{equation*}
  S=\{\ve{s_0},\ve{s_1},\dots,\ve{s_n}\} \subset \mE^d
\end{equation*}
of $n$ sites in the euclidean space of dimension $d$, and we build a
set of $n$ Voronoi \emph{cells}\footnote{$2^{\mE^d}$ is the power set of
  $\mE^d$, the set of all the subsets of $\mE^d$.}
\begin{equation*}
  Vor(S)=\{V(\ve{s_0}),\dots,V(\ve{s_n})\}\subset 2^{\mE^d}
\end{equation*}
such that
\begin{equation*}
  V(\ve{\ve{s_i}})=\{\ve{p}\in\mE^d\ :\
  \norm{\ve{p}-\ve{s_i}}<\norm{\ve{p}-\ve{s_j}}\ \forall \ve{s_j}\neq\ve{s_i}\}
\end{equation*}
is the set of the points in $\mE^d$ closer to $\ve{s_i}$ than to any
other site.

\image{voronoi.eps}{Example of a \ac{VD}, dashed lines are infinite edges.}{fig:voronoi}
\Cref{fig:voronoi} is an example of the \ac{VD} built on some random
sites, on the figure the dashed lines are edges that go to infinite.

\image{fortune.png}{Fortune's algorithm execution}{fig:fortune}
\index{Fortune's algorithm}The most important algorithm for
calculating \acp{VD} is the \emph{Fortune's}
\emph{sweeping line} algorithm that builds the diagram in $\bigO(n\log
n)$ and it is optimal. The algorithm involves building $Vor(S)$
incrementally while sweeping the space, see \cref{fig:fortune}. Every
time that the sweeping
line finds a site the algorithm creates a parabola using the site as
focus and the sweeping line as directrix. Such parabolas, or better
the arcs between each intersection of them, constitute
the \emph{beach line}. A parabola disappears from the scene when the
associated
arc vanish. The evolution of the intersection points on the beach line
constitutes the edges of the \ac{VD}, and each point where an arc of
the beach line disappears constitutes a vertex of the \ac{VD}.
Refer to \cite{deberg} and
\cite{fortune} for details about the Fortune's algorithm.

One property of \acp{VD} is that $V(\ve{s_i})$ can be a closed or an
open area - i.e. the edges of the cells can be infinite - it is important
to keep this in mind if we want to interpret $Vor(S)$ as a graph. On
that case the graph will have edges that go to infinite. We call such
graph $G(Vor(S))$.

Another property is that if we have $d+1$ sites $\ve{s_0'},\dots,\ve{s_d'}$ that lay on the
surface of a $(d-1)$-sphere\footnote{A circumference in $2$-dimensional
  space, a sphere in $3$-dimensional space, an hypersphere in
  $n$-dimensional space with $n\ge 3$.} that doesn't have any other site on
the interior, then the center point of the $(d-1)$-sphere is the
vertex shared only between the $d+1$ cells
$V(\ve{s_0'}),\dots,V(\ve{s_d'})$ \cite{deberg}. This is not true for
less than
$d+1$ sites on a
$(d-1)$-sphere because they are not enough to define it univocally,
but is possible to have $n>d+1$ sites on a $(d-1)$-sphere, on that
case the center of such $(d-1)$-sphere is the shared vertex of the
cells corresponding to the $n$ sites. This is important for reasoning
about the topography of $G(Vor(S))$ because if we allow more than
$d+1$ sites on a $(d-1)$-sphere then the maximum degree
$\Delta(G(Vor(S)))$ of the graph can be arbitrarily big (until to the
number of vertices). However the fact that we work with coordinates in
$\mE^d$ legitimizes the restriction\footnote{We can also relax this
  restriction and in case create multiple nodes connected by zero-distance
  edges on the graph.} of not
allowing more than $d+1$ sites on an $(d-1)$-sphere, limiting
$\Delta(G(Vor(S)))$ to $d+1$.

\section{Statistical methods}\label{sec:statisticalMethods}
In this section we make a short introduction to \acf{MCM}
\cite{metropolis}\cite{sobol}\cite{newman} and \acf{SA} \cite{kirkpatrick}\cite{ho},
two statistical methods for calculating unknown quantities and finding
minimum of functions. Also we introduce \acf{LR} \cite{benjamin} a
method for transforming a constrained optimization problem in an
unconstrained optimization problem increasing the state space
dimension.
\subsection{Notes on probabilities}
\subsubsection{\ac{PE}}\index{\acf{PE}}
For a random variable $X$ normally distributed with
\begin{itemize}
\item mean $\mu$;
\item variance $\sigma$;
\end{itemize}
for
\begin{equation*}
  r=0.6745\sigma
\end{equation*}
we have that
\begin{equation*}
  \prob{\abs{X-\mu}<r} = \prob{\abs{X-\mu}>r} = 0.5
\end{equation*}
So values of $X$ that deviate from $\mu$ approximately by $r$ have the
same probability, and $r$ identifies the most \ac{PE} in a
normal distribution.
\subsubsection{\acf{PCLT}}\index{\acf{PCLT}}
Consider $N$ independent and \emph{identically-distributed} random variables
$X_1,X_2,\dots,X_N$, with same mean and same variance
\begin{eqnarray*}
  \expected{X_1}=\expected{X_2}=\dots=\expected{X_N}&=&m\\
  \variance{X_1}=\variance{X_2}=\dots=\variance{X_N}&=&b^2.
\end{eqnarray*}

Consider the sum of those random variables:
\begin{equation*}
  Y = X_1+X_2+\cdots+X_N
\end{equation*}
we have that
\begin{eqnarray*}
  \expected{Y}&=&\expected{X_1+X_2+\cdots+X_N}=Nm\\
  \variance{Y}&=&\variance{X_1+X_2+\cdots+X_N}=Nb^2.
\end{eqnarray*}

Consider now a normally distributed random variable $Z$ with
parameters:
\begin{eqnarray*}
  \mu&=&Nm\\
  \sigma&=&b\sqrt{N}
\end{eqnarray*}
with \ac{PDF} $p_Z(x)$.

The \emph{\ac{PCLT}} affirms that for $N$ big
enough, and for every interval $(x_1,x_2)$:
\begin{equation}\label{eq:tcl}
  \prob{x_1<Y<x_2}\approx\int_{x_1}^{x_2}p_Z(x) \md x.
\end{equation}
So the sum of an elevate number of identically-distributed random
variables is a random variable with normal distribution with mean $Nm$
and variance $Nb^2$ even if
$X_1,X_2,\dots,X_N$ aren't normally distributed.

\subsection{\acf{MCM}}\index{\acf{MCM}}
Suppose that you need to calculate an unknown quantity $m$, you need
to find a random variable $X$ such that:
\begin{equation*}
  \expected{X} = m.
\end{equation*}
If you have such distribution with variance:
\begin{equation*}
  \variance{X} = b^2
\end{equation*}
it is possible to formalize the following passages.

Consider $N$ random variables $X_1,X_2,\dots,X_N$ that have
distribution identical to the distribution of $X$. For the \ac{PCLT}
\cref{eq:tcl} we have that, for $N$ big enough
\begin{equation*}
  Y=X_1+X_2+\cdots+X_N
\end{equation*}
is normally distributed with parameters
\begin{eqnarray*}
  \mu &=& Nm\\
  \sigma&=&b\sqrt{N}.
\end{eqnarray*}

For the \emph{three sigma rule} we have that:
\begin{equation*}
  \prob{\mu-3\sigma < Y <\mu +3\sigma}\approx 0.997
\end{equation*}
that is
\begin{equation*}
  \prob{Nm-3b\sqrt{N} < Y < Nm+3b\sqrt{N}}\approx 0.997
\end{equation*}
dividing by $N$:
\begin{equation*}
  \prob{m-\frac{3b}{\sqrt{N}} < \frac{Y}{N} <
    m+\frac{3b}{\sqrt{N}}}\approx 0.997
\end{equation*}
that is
\begin{equation*}
  \prob{\abs{\frac{Y}{N}-m} <\frac{3b}{\sqrt{N}}}\approx 0.997
\end{equation*}
and so:
\begin{equation}\label{eq:mc}
  \prob{\abs{\frac{1}{N}\sum_{i=1}^NX_i-m} <\frac{3b}{\sqrt{N}}}\approx 0.997.
\end{equation}

\Cref{eq:mc} asserts that, if you extract a sample for each random
variable $X_i$, the arithmetic mean of those values is approximately
equal to $m$. Moreover the error of such approximation is equal to
$3b/\sqrt{N}$, that tend to $0$ increasing $N$. It is also possible to
further 
reduce the uncertainty ($1-0.997=0.003$) by increasing the number $k$ of
sigma used for the
approximation and evaluating the error $kb/\sqrt{N}$.

In practice, since the random variables $X_i$ have the same
distribution of $X$, it is sufficient to extract $N$ samples from $X$ for
reaching to the same conclusions.

The \ac{MCM} is constituted by the following procedure, to be adapted
according to the problems:
\begin{enumerate}
\item find the distribution $X$ having desired quantity $m$ as mean
  value and $b^2$ as variance;
\item extract $N$ samples from $X$, with $N$ big enough to have an
  error small as desired;
\item the arithmetic mean of those $N$ samples is the approximation of
  the desired value $m$.
\end{enumerate}
Basically we transformed the problem from \emph{calculate $m$} to \emph{find
  the distribution $X$}, or anyway the $N$ samples distributed
accordingly to $X$.

If we want to characterize more in detail the error committed taking
$N$ samples, we can recur to \ac{PE}. If we set $k=0.6745$ then we
have that
\begin{equation*}
  \prob{\abs{\frac{1}{N}\sum_{i=1}^NX_i-m} <\frac{0.6745\cdot b}{\sqrt{N}}}\approx 0.5
\end{equation*}
and so
\begin{equation*}
  r_N = \frac{0.6745\cdot b}{\sqrt{N}}
\end{equation*}
indicates how much the value $\frac{1}{N}\sum_{i=1}^NX_i$ deviates
from the desired value $m$. Such value characterize the absolute error
\begin{equation*}
\abs{\frac{1}{N}\sum_{i=1}^NX_i-m}  
\end{equation*}
committed taking $N$ samples.

\ac{MCM} is useful to simulate events that have an high degree of
uncertainty in the inputs or an high degree of liberty in the
state. For instance integrating numerically a function with many
dimensions or \ac{SA}.

\subsection{\acf{SA}}\index{\acf{SA}}
The \ac{SA} is a method used for finding the global optimum - i.e. the
maximum or the minimum - of a function. It is inspired by a method
used in metallurgy that consists in heating and then cooling slowly a
material for augmenting the dimensions of the crystals and improving
the chemico-physical properties. The function to be optimized can be
defined in a multiple-dimensional space.

\subsubsection{Statistical thermodynamic}
For describing the base principles of statistical thermodynamic we
consider an example. In a one-dimensional lattice every point
is a particle with a value of spin that can be \emph{up} or
\emph{down}. If the lattice has $N$ points then the system can be in
$2^N$ different configurations, where each one of those configurations
corresponds to a value of energy, for instance:
\begin{equation*}
  E=B(n_+-n_-)
\end{equation*}
where $B$ is some constant, $n_+$ is the number of particles with spin
\emph{up} and $n_-$ is the number of particles with spin \emph{down}.

The probability $P(\sigma)$ of finding the system in a certain
configuration $\sigma$ is given by the distribution of
\emph{Boltzmann-Gibbs}:
\begin{equation}\label{eq:distBoltz}
  P(\sigma) = C \me^{-E_\sigma/T}
\end{equation}
where $E_\sigma$ is the energy of the configuration, $T$ is the
temperature\footnote{The real Boltzmann-Gibbs distribution is
  $P(\sigma) = C \me^{-E_\sigma/kT}$ where $k$ is the \emph{Boltzmann
    constant} and $T$ is the thermodynamic temperature, but for the
  example the temperature is a parameter not correlated to the
  physical world, so is possible to ignore $k$.} and $C$ is a
normalization constant.

The average energy of the system is then:
\begin{eqnarray*}
  \bar{E} &=& \frac{\sum_\sigma E_\sigma P(\sigma)}{\sum_\sigma
    P(\sigma)}\\
  &=& \frac{\sum_\sigma E_\sigma \me^{-E_\sigma/T}}{\sum_\sigma \me^{-E_\sigma/T}}.
\end{eqnarray*}
The computation of the value of $\bar{E}$ can be difficult with an
high number of 
states, but it is possible to create a \ac{MCM} simulating the random
fluctuation between the states such that the distribution given by
\cref{eq:distBoltz} is respected. Starting from an arbitrary initial
configuration, after a certain number of \emph{Monte Carlo trials},
the method converges to the equilibrium status $\bar{E}$ and it
continues
to fluctuate around it. \ac{SA} is a method of this kind.

\subsubsection{\acf{SA} algorithm}\index{\acf{SA}!algorithm}
\ac{SA} operates on a system starting from a certain initial state
$s_0$, then it executes a series of iterations where a
neighbour of the state is evaluated and, with a certain distribution
of probability, the system is moved in the new state or not.

\begin{algo}{\acf{SA}}{alg:sa}
  \Function{anneal}{$s_0$}
  \State $s\Ass s_0$
  \For{$k\Ass 0,kMax$}
  \State $T\Ass temp(\frac{k}{kMax})$
  \State $sNew\Ass neighbour(s)$
  \If{$uniform(0,1)<P(E(s), E(sNew), T)$}
  \State $s\Ass sNew$
  \EndIf
  \EndFor
  \State\Return $s$
  \EndFunction
\end{algo}
A possible algorithm for a \ac{SA} method is \cref{alg:sa}. $s_0$ is
the initial state; $temp$ is the function that assigns a
temperature based on the current iteration number such that for low
$k$ the returned temperature is high and for high $k$ the returned temperature
is low; $neighbour$ is the function that returns a random neighbour of
the current state; $uniform$ returns an uniformly-randomly chosen
number in $[0,1]$; $P$ is the distribution of accepting probability,
that depends on the energy of the current state, on the energy of the
neighbour, and on the current temperature. In case of acceptation the
neighbour becomes the current state and the process continues.

The relation with the statistical thermodynamic is that $P$ is chosen
such that
\cref{eq:distBoltz} holds\footnote{A similar distribution is enough.},
moreover $temp$ returns decreasing values of
temperature with the succession of iterations, this is why the
comparison with the metallurgy annealing.

Initially $P$ was chosen such that
\begin{equation*}
  P(E(s), E(sNew), T)=
  \begin{cases}
    1,& \text{if }E(sNew)<E(s)\\
    \me^{-(E(sNew)-E(s))/T},& \text{otherwise}
  \end{cases}
\end{equation*}
but this isn't strictly necessary for developing a \ac{SA} method.

\subsection{\acf{LR}}\label{sec:lagrangianRelaxation}\index{\acl{LR}}
A general constrained discrete optimization problem can be expressed in
the form:
\begin{equation}\label[problem]{eq:opt}
\begin{aligned}
& \underset{x}{\text{minimize}}
& & f(x) \\
& \text{subject to}
& & g(x)=0.
\end{aligned}
\end{equation}
where $x\in X$ is the state of the system in a discrete space $X$, $f(x)$
is the function to
minimize, and $g(x)=0$ is the constraint. The functions can also be
in a multidimensional discrete space, in that case the $x$ is a vector
$\ve{x}=(x_1,\dots,x_n)$ of variables.

For solving this class
of problems a \emph{Lagrange relaxation} method can be used, it
augments the variable space $X$ by a \emph{Lagrange multiplier} space
$\Lambda$ of dimension equal to the number of constraints - one in the
\cref{eq:opt}.

The \emph{generalized discrete Lagrangian
  function} corresponding to the \cref{eq:opt} is:
\begin{equation}\label{eq:lagrangianFun}
  L_d(x,\lambda)=f(x)+\lambda H(g(x))
\end{equation}
where $\lambda$ is a variable in $\Lambda$, and if the dimension of
$\Lambda$ is more than one $\lambda$ must be transposed in
~\cref{eq:lagrangianFun}. $H(x)$ is a non negative function
with the property that $H(0)=0$. the purpose of it is to transform $g(x)$ in
a non negative function, if $g(x)$ isn't already not negative. For
instance can be $H(g(x))=|g(x)|$ or $H(g(x))=g^2(x)$.

Under the previous assumptions the set of \emph{local minima}
in \cref{eq:opt} - that respect the constraints -  coincides
with the set of \emph{discrete saddle point}
in the augmented space. A point $(x^*,\lambda^*)$ is a discrete saddle
point if:
\begin{equation*}
  L_d(x^*,\lambda)\leq L_d(x^*,\lambda^*)\leq L_d(x,\lambda^*)
\end{equation*}
for all $x\in\mathcal{N}(x^*)$ and for all $\lambda\in\Lambda$, where
$\mathcal{N}(x^*)$ is the set of all neighbours of $x^*$.

For resolving the optimization \cref{eq:opt} it is necessary to
calculate all the discrete saddle points $(x^*,\lambda^*)$ on the surface
represented by
\cref{eq:lagrangianFun} using some
optimization method (i.e. simulated annealing). Then it is chosen
the one that minimize $f(x^*)$, $x^*$ is then the desired minimum.

\section{Intersections in space}\label{sec:intersections}\index{3-D geometry}
In the project we need to deal with three kinds of collision detection
methods in 3-dimension euclidean space:
\begin{itemize}
\item we need to check if a point is in or out
  of a convex polyhedron;
\item we need to check if a segment intersects a triangle;
\item and we need to check if a
  triangle intersects another triangle.
\end{itemize}

\subsection{Point inside convex polyhedron in 3-d space}\index{3-D
  geometry!point inside polyhedron}
For testing if a point $\ve{p}$ is inside a convex polyhedron $V$ with
vertices $\ve{v_1},\ve{v_2},\dots,\ve{v_n}$ we used a method that rely
on convex hulls \cite{deberg}\cite{schneider}.

\begin{algo}{Check if point $\ve{p}$ is inside convex polyhedron $V$}{alg:pointInPoly}
  \Function{isPointInPolyhedron}{$\ve{p}$, $V$}
  \State $inside\Ass\True$
  \State $C\Ass convexHullVertices([\ve{p},\ve{v_1},\ve{v_2},\dots,\ve{v_n}])$
  \ForAll{$\ve{c}\in C$}\label{ln:pointInPolyFor}
  \If{$c=p$}
  \State $inside\Ass\False$
  \State \Break
  \EndIf
  \EndFor
  \State\Return $inside$
  \EndFunction
\end{algo}
As in the \cref{alg:pointInPoly} first we build the convex hull of all
the vertices of $V$ plus the point $\ve{p}$, then we check if $\ve{p}$
is on the convex hull or not. If $\ve{p}$ is on the convex hull that
means that $\ve{p}$ is external\footnote{Or $\ve{p}$ coincides with a
  vertex of $V$.} to $V$ because we have extended the convex hull
formed by the vertices of $V$. Otherwise this means that $\ve{p}$ is
inside $V$.

The cost of this algorithm is
\begin{equation*}
  \bigO(n\log n)
\end{equation*}
where $n$ is the number of vertices of $V$, because the cost for
constructing the convex hull is \cite{deberg} $\bigO(n\log n)$, and
then we have another negligible term $\bigO(n)$ for the cycle on \cref{ln:pointInPolyFor}.

\subsection{Segment-triangle in 3-d space}\label{sec:intersectionST}\index{3-D
  geometry!segment-triangle intersection}\index{intersections!segment-triangle}
We need to deal with the intersection between a segment
$S=\overline{\ve{a_2}\ve{b_2}}$ and a triangle
$T=\triangle\ve{a_1}\ve{b_1}\ve{c_1}$. $S$ and $T$ can be in one of
the cases:
\begin{enumerate}[label=\textbf{case \arabic*}]
\item\label[void]{en:ist:nc:ni} $S$ and $T$ do not intersect and the plane containing
  $T$ is not in the sheaf of planes generated by the line
  containing $S$;
\item\label[void]{en:ist:c:ni} $S$ and $T$ do not intersect and the plane containing
  $T$ is in the sheaf of planes generated by the line
  containing $S$;
\item\label[void]{en:ist:nc:i} $S$ and $T$ intersect in only one point and the plane containing
  $T$ is not in the sheaf of planes generated by the line
  containing $S$;
\item\label[void]{en:ist:c:i} $S$ and $T$ intersect in one or infinite points and the plane
  containing $T$ is in the sheaf of planes generated by the line
  containing $S$.
\end{enumerate}
The discriminant between the cases is that if $S$ and $T$ are
intersecting and if $S$ and $T$ are coplanar. The cases are resumed in
\cref{tab:intersectSegmentTriang}.
\begin{table}
  \centering
  \begin{tabular}{l|cc}
    &not coplanar&coplanar\\
    \hline
    not intersect& \cref{en:ist:nc:ni} & \cref{en:ist:c:ni}\\
    intersect& \cref{en:ist:nc:i} & \cref{en:ist:c:i}\\
  \end{tabular}
  \caption{Relations between $S$ and $T$}
  \label{tab:intersectSegmentTriang}
\end{table}

\begin{myfig}{Example intersection between a segment
    $\overline{\ve{a_2}\ve{b_2}}$ and a triangle $\triangle \ve{a_1}\ve{b_1}\ve{c_1}$.}{fig:segmentTriangleIntersection}
  \begin{tikzpicture}
    \coordinate (A1) at (0,0);
    \coordinate (B1) at (4,5);
    \coordinate (C1) at (8,1);

    \coordinate (cut1) at (barycentric cs:A1=0.8,B1=0.,C1=0.2);
    \coordinate (cut2) at (barycentric cs:A1=0.,B1=0.8,C1=0.2);

    \coordinate (A2) at (2,5);    
    \coordinate (B2) at (4.5,1.5);

    \coordinate (X) at (intersection of cut1--cut2 and B2--A2);

    \draw[poly] (A1) -- (C1) -- (B1) -- (A1);
    \draw[poly] (A2) -- (X);
    \draw[polyTract] (B2) -- (X);

    \foreach \p in {A1,B1,C1,A2,B2}
    \filldraw[vertex] (\p) circle (2pt);

    \filldraw[intersection] (X) circle (2pt);

    \node[below=0.5em] at (A1) {$\ve{a_1}$};
    \node[above=0.5em] at (B1) {$\ve{b_1}$};
    \node[below=0.5em] at (C1) {$\ve{c_1}$};
    \node[above=0.5em] at (A2) {$\ve{a_2}$};
    \node[right=0.5em] at (B2) {$\ve{b_2}$};
    \node[right=0.2em] at (X) {$\ve{x}$};
  \end{tikzpicture}
\end{myfig}
In \cref{fig:segmentTriangleIntersection} is visible a
\cref{en:ist:nc:i} relation where there is intersection in only one
point $\ve{x}$. For calculating if there is intersection we can solve
four equation in four unknowns \cite{schneider} where we look for
$\ve{x}$ searching a point that is a convex linear
combination of $\ve{a_2}$ and $\ve{b_2}$ and at the same time a convex
linear combination of $\ve{a_1}$,
$\ve{b_1}$ and $\ve{c_1}$. In other words when there is a collision,
then there is a solution for the unknowns
$\alpha$, $\beta$, $\gamma$, $\delta$, $\zeta$ of the system
\begin{equation}\label[system]{eq:segmentTriangleIntersection1}
  \begin{cases}
    \alpha \ve{a_2} + \beta\ve{b_2}=\gamma\ve{a_1}+\delta\ve{b_1}+\zeta\ve{c_1} \\
    \alpha + \beta = 1\\
    \gamma + \delta +\zeta=1
  \end{cases}
\end{equation}
with the further conditions
\begin{equation}\label[system]{eq:segmentTriangleIntersection1c}
  \begin{cases}
    \alpha \ge 0\\
    \beta \ge 0\\
    \gamma \ge 0\\
    \delta \ge 0\\
    \zeta \ge 0.
  \end{cases}
\end{equation}
Note that the first equation of
\cref{eq:segmentTriangleIntersection1} has vectorial coefficients
$\ve{a_2}$, $\ve{b_2}$, $\ve{a_1}$, $\ve{b_1}$, $\ve{c_1}$, so the
system is of five unknowns in five equations. If
\Cref{eq:segmentTriangleIntersection1} has just one solution then we are
on \cref{en:ist:nc:ni} or \cref{en:ist:nc:i}, if it has infinite
solution then we are on \cref{en:ist:c:ni} or \cref{en:ist:c:i}, if
it has not solution then $S$ or $T$ are degenerated. If
\cref{eq:segmentTriangleIntersection1c} is satisfied for the solution(s)
of \cref{eq:segmentTriangleIntersection1} then we are in
\cref{en:ist:nc:i} or \cref{en:ist:c:i}, if is not satisfied then we
are in \cref{en:ist:nc:ni} or \cref{en:ist:c:ni}.

We are interested in finding only \cref{en:ist:nc:i} collisions because for
simplicity we consider the special case of a segment that lays on the
surface of a triangle as not colliding with it, and for coherence we
restrict the conditions of \cref{eq:segmentTriangleIntersection1c} to
\begin{equation}
  \begin{cases}
    \alpha > 0\\
    \beta > 0\\
    \gamma > 0\\
    \delta > 0\\
    \zeta > 0.
  \end{cases}
\end{equation}

\Cref{eq:segmentTriangleIntersection1} can be simplified in the three
equations 
\begin{equation}\label[system]{eq:segmentTriangleIntersection2}
  \begin{cases}
    \alpha \ve{a_2} + (1-\alpha)\ve{b_2}=\gamma\ve{a_1}+\delta\ve{b_1}+(1-(\gamma+\delta))\ve{c_1} \\
  \end{cases}
\end{equation}
in the unknowns $\alpha$, $\gamma$ and $\delta$ with the relative conditions
\begin{equation}\label[system]{eq:segmentTriangleIntersection2c}
  \begin{cases}
    \alpha > 0\\
    \alpha < 1\\
    \gamma > 0\\
    \delta > 0\\
    \gamma+\delta<1.
  \end{cases}
\end{equation}

\begin{algo}{Find intersection between segment $S$ and triangle $T$}{alg:intersectSegmentTriangle}
  \Function{intersect}{$S, T$}
  \State $intersect\Ass\False$
  \State $coordinates\Ass\emptyset$
  \If{$(\alpha,\gamma,\delta)\Ass
    solve($\cref{eq:segmentTriangleIntersection2}$)$}\label{ln:intersectSegmentTriangleIf}
  \If{$satisfy($\cref{eq:segmentTriangleIntersection2c}$)$}
  \State $intersect\Ass\True$
  \State $coordinates\Ass(\gamma, \delta, 1-(\gamma+\delta))$
  \EndIf
  \EndIf
  \State\Return $(intersect, coordinates)$
  \EndFunction
\end{algo}
The \cref{alg:intersectSegmentTriangle} for calculating the
intersection consists basically in solving
\cref{eq:segmentTriangleIntersection2} with the parameters $\ve{a_2}$,
$\ve{b_2}$, $\ve{a_1}$, $\ve{b_1}$ and $\ve{c_1}$ from $S$ and $T$;
and then in checking if the solution is admissible. The condition of
\cref{ln:intersectSegmentTriangleIf} is $\True$ if
\cref{eq:segmentTriangleIntersection2} has solution and if that
is unique.

We have also the positive secondary effect that from the solution
$(\alpha,\gamma,\delta)$ of \cref{eq:segmentTriangleIntersection2} we
can extract the barycentric coordinates $(\gamma, \delta,
1-(\gamma+\delta))$ of the intersection point $\ve{x}$ on the system of
the vertices $\ve{a_1}$, $\ve{b_1}$, $\ve{c_1}$ of $T$.

\subsection{Triangle-triangle in 3-d space}\label{sec:intersectionsTriangleTriangle}\index{3-D
  geometry!triangle-triangle intersection}\index{intersections!triangle-triangle}
We are interested in detecting collisions between two triangles
$T_1=\triangle \ve{a_1}\ve{b_1}\ve{c_1}$ and $T_2=\triangle\ve{a_2}\ve{b_2}\ve{c_2}$ in 3-dimensional
space. First of all consider the coplanarity relation between the two
triangles, we have the cases:
\begin{enumerate}[label=\textbf{case \arabic*}]
\item\label[void]{en:itt:c} $T_1$ and $T_2$ are contained by the same plane;
\item\label[void]{en:itt:nc} $T_1$ and $T_2$ are contained by different planes.
\end{enumerate}
To simplify the problem we decided - analogously to the case of
intersection between segment and triangle - that when we are on \cref{en:itt:c}
we consider $T_1$ and $T_2$ not intersecting in any case, also if from a
geometrical point of view they share points. After this premise we can
assert that the possible relation between $T_1$ and $T_2$ can be
exclusively one
of the types \cite{schneider}:
\begin{enumerate}[label=\textbf{type \arabic*}, start=0]
\item\label[void]{en:itt:i0} $T_1$ and $T_2$ do not intersect;
\item\label[void]{en:itt:i1} two edges of $T_1$ intersect the plane
  section delimited by $T_2$, or vice versa;
\item\label[void]{en:itt:i2} one edge of $T_1$ intersects the plane
  section delimited by $T_2$ and one edge of $T_2$ intersects the
  plane section delimited by $T_1$.
\end{enumerate}

\begin{myfig}{Example of \cref{en:itt:i1} intersection between a triangle $T_1=\triangle \ve{a_1}\ve{b_1}\ve{c_1}$
    and another triangle $T_2=\triangle \ve{a_2}\ve{b_2}\ve{c_2}$.}{fig:trianglesIntersection1}
  \begin{tikzpicture}
    \coordinate (A1) at (0,0);
    \coordinate (B1) at (4,5);
    \coordinate (C1) at (8,1);

    \coordinate (cut1) at (barycentric cs:A1=0.8,B1=0.,C1=0.2);
    \coordinate (cut2) at (barycentric cs:A1=0.,B1=0.8,C1=0.2);

    \coordinate (A2) at (0,3);
    \coordinate (B2) at (2,5);    
    \coordinate (C2) at (4.5,1.5);

    \coordinate (Xs1) at (intersection of A1--B1 and C2--A2);
    \coordinate (Xs2) at (intersection of A1--B1 and C2--B2);
    \coordinate (X1) at (intersection of cut1--cut2 and C2--A2);
    \coordinate (X2) at (intersection of cut1--cut2 and C2--B2);

    \draw[poly] (Xs1) -- (A1) -- (C1) -- (B1) -- (Xs2);
    \draw[polyTract] (Xs1) -- (Xs2);
    \draw[poly] (X1) -- (A2) -- (B2) -- (X2);
    \draw[polyTract] (X1) -- (C2) -- (X2);
    \draw[cutting] (X1) -- (X2);

    \foreach \p in {A1,B1,C1,A2,B2,C2}
    \filldraw[vertex] (\p) circle (2pt);

    \filldraw[intersection] (X1) circle (2pt);
    \filldraw[intersection] (X2) circle (2pt);

    \node[below=0.5em] at (A1) {$\ve{a_1}$};
    \node[above=0.5em] at (B1) {$\ve{b_1}$};
    \node[below=0.5em] at (C1) {$\ve{c_1}$};
    \node[left=0.5em] at (A2) {$\ve{a_2}$};
    \node[above=0.5em] at (B2) {$\ve{b_2}$};
    \node[right=0.5em] at (C2) {$\ve{c_2}$};
    \node[below=0.2em] at (X1) {$\ve{x_1}$};
    \node[right=0.2em] at (X2) {$\ve{x_2}$};
  \end{tikzpicture}
\end{myfig}
\begin{myfig}{Example of \cref{en:itt:i2} intersection between a triangle $T_1=\triangle \ve{a_1}\ve{b_1}\ve{c_1}$
    and another triangle $T_2=\triangle \ve{a_2}\ve{b_2}\ve{c_2}$.}{fig:trianglesIntersection2}
  \begin{tikzpicture}
    \coordinate (A1) at (0,0);
    \coordinate (B1) at (4,5);
    \coordinate (C1) at (8,1);

    \coordinate (cut) at (barycentric cs:A1=0.8,B1=0.,C1=0.2);
    \coordinate (X2) at (barycentric cs:A1=0.,B1=0.8,C1=0.2);

    \coordinate (A2) at (0,6);
    \coordinate (B2) at (8,7);    
    \coordinate (C2) at (4.5,1.5);

    \coordinate (Xs1) at (intersection of A1--B1 and C2--A2);
    \coordinate (Xs2) at (intersection of C1--B1 and C2--B2);
    \coordinate (X1) at (intersection of cut--X2 and C2--A2);

    \draw[poly] (Xs1) -- (A1) -- (C1) -- (X2);
    \draw[polyTract] (Xs1) -- (B1) -- (X2);
    \draw[poly] (X1) -- (A2) -- (B2) -- (Xs2);
    \draw[polyTract] (X1) -- (C2) -- (Xs2);
    \draw[cutting] (X1) -- (X2);

    \foreach \p in {A1,B1,C1,A2,B2,C2}
    \filldraw[vertex] (\p) circle (2pt);

    \filldraw[intersection] (X1) circle (2pt);
    \filldraw[intersection] (X2) circle (2pt);

    \node[below=0.5em] at (A1) {$\ve{a_1}$};
    \node[above=0.5em] at (B1) {$\ve{b_1}$};
    \node[below=0.5em] at (C1) {$\ve{c_1}$};
    \node[left=0.5em] at (A2) {$\ve{a_2}$};
    \node[above=0.5em] at (B2) {$\ve{b_2}$};
    \node[right=0.5em] at (C2) {$\ve{c_2}$};
    \node[below=0.2em] at (X1) {$\ve{x_1}$};
    \node[right=0.2em] at (X2) {$\ve{x_2}$};
  \end{tikzpicture}
\end{myfig}
\begin{algo}{Find intersection between triangle $T_1$ and triangle $T_2$}{alg:intersectTriangles}
  \Function{intersect}{$T_1=(\ve{a_1},\ve{b_1},\ve{c_1}),\ T_2=(\ve{a_2},\ve{b_2},\ve{c_2})$}
  \ForAll{$S\in \{\overline{\ve{a_1}\ve{b_1}},\ \overline{\ve{b_1}\ve{c_1}},\ \overline{\ve{c_1}\ve{a_1}}\}$}
  \If{$intersect(S,T_2)$}\label{ln:intersectTriangles1}
  \State\Return $\True$
  \EndIf
  \EndFor
  \ForAll{$S\in \{\overline{\ve{a_2}\ve{b_2}},\ \overline{\ve{b_2}\ve{c_2}},\ \overline{\ve{c_2}\ve{a_2}}\}$}
  \If{$intersect(S,T_1)$}\label{ln:intersectTriangles2}
  \State\Return $\True$
  \EndIf
  \EndFor
  \State\Return $\False$
  \EndFunction
\end{algo}
On \cref{fig:trianglesIntersection1} and
\cref{fig:trianglesIntersection2} we can see two examples of
\cref{en:itt:i1} and \cref{en:itt:i2}. For finding if $T_1$ and $T_2$
intersect we need to check if every edge of $T_1$ intersects $T_2$
and every edge of $T_2$ intersects $T_1$. If we found at least one
edge that intersects with one triangle then $T_1$ and $T_2$
intersect. \Cref{alg:intersectTriangles} executes such check, the function
$intersect$ on \cref{ln:intersectTriangles1} and
\cref{ln:intersectTriangles2} is the intersection check between a segment
and a triangle described on \cref{alg:intersectSegmentTriangle}.

\end{document}

%%% Local Variables:
%%% mode: latex
%%% TeX-master: "../dissertation"
%%% End:
