\documentclass[dissertation.tex]{subfiles}
\begin{document}
\chapter{Code structure}\label{cha:codeStructure}
We designed the code with an \ac{OOP} methodology in Python 3
(\url{https://www.python.org/}). A versatile language with a strong
appeal on
scientific community, easy to learn and with an increasing active
community of developers behind. We
relied on SciPy (\url{https://www.scipy.org/}) and NumPy
(\url{http://www.numpy.org/}) libraries for taking care of
different numerical methods. Furthermore we used NetworkX library
(\url{http://networkx.github.io/}) to represent graphs and
to route in them. Regarding the graphic output we used \ac{VTK}
(\url{http://www.vtk.org/}) bindings in Python.

\imagev{uml.eps}{Excerpt \acs{UML} of the project}{fig:uml}
The main class is \pp{Voronizator}, it maintains the status of the
scene and provides all the methods for the interface with users.
\begin{itemize}
\item \pp{addBoundingBox} is used for adding a bounding box at specified
  coordinates to the scene.
\item \pp{addPolyhedron} is used for adding a new
  obstacle to the scene, it required a \pp{Polyhedron} object as
  argument.
\item \pp{setPolyhedronsSites} add the sites for the \ac{VD} to the scene.
\item The method \pp{makeVoroGraph} is used for creating the graphs
  $G$ and $G_t$ using the algorithms described in \cref{sec:baseGraph}
  and \cref{sec:trigraph}.
\item \pp{setAdaptivePartition} and \pp{setBsplineDegree} are used for
  selecting between uniform or adaptive knot partition, and for choosing
  the desired degree of the curve.
\item \pp{extractXmlTree} and \pp{importXmlTree} are used for saving
  and restoring the scene in \acs{XML} format.
\item All the other methods \pp{plot*} are used for drawing the
  different elements of the scene. They require a plotter as
  argument.
\end{itemize}

The class \pp{Plotter} provides the interface for drawing all the
necessary elements using \ac{VTK}.

The class \pp{Polyhedron} represents a single obstacle (that can also
be one of the two subclasses \pp{ConvexHull} and \pp{Tetrahedron}). it
provides all the necessary methods for performing geometry checks of
point inclusion and intersection with segments, triangles, and other
obstacles.

The class \pp{Path} represents the control polygon of the curve. It
provides the methods \pp{addNAlignedVertexes} and \pp{simplify} that
perform respectively the degree increase (\cref{sec:degreeInc}) and
the post processing (\cref{sec:postPro}). The method \pp{clean} is
necessary for the second solution described in
\cref{sec:inter2}. Furthermore this class provides also the
functionality for optimizing the curve using the \ac{SA}
(\cref{sec:inter3}) with the method \pp{anneal}.

Furthermore we provide scripts for the creation of random scenes and
for the execution of the different methods.

\end{document}

%%% Local Variables:
%%% mode: latex
%%% TeX-master: "../dissertation"
%%% End:
