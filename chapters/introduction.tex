\documentclass[dissertation.tex]{subfiles}
\begin{document}
\chapter{Introduction}
The design of \emph{motion planning} strategies plays a fundamental
role in different applications, from robotics to scientific
visualization \cite{giannelli}. \emph{Path planning} problem is
narrower, it includes the identification of paths that do not
intersect any obstacle \cite{giannelli}. Such paths are designed using
\bs curves, a reference standard in \ac{CAD} and \ac{CAGD}
\cite{hughes}\cite{foley}.

The considered topic is highly cross-disciplinary. In fact we designed
this project with the help of an extended set of
competencies acquired following the courses. We applied notions of
\emph{linear algebra} for the collision checks; \emph{numerical
  analysis} for the design of the curves; \emph{probability},
\emph{graph theory} and \emph{algorithm theory} for the design of the
algorithms; and finally \emph{theoretical computer science} for the
analysis of the costs.

We focused on finding a trade-off between having a short curve, a
fairly efficient computation, and a smooth curve. We explored
different solutions with different collocations on the trade-off
triangle and with different qualitative effects on the curve.
\end{document}

%%% Local Variables:
%%% mode: latex
%%% TeX-master: "../dissertation"
%%% End:
