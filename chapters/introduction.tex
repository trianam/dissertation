\documentclass[dissertation.tex]{subfiles}
\begin{document}
\chapter{Introduction}
The design of \emph{motion planning} strategies plays a fundamental
role in different applications, from robotics to scientific
visualization. \emph{Path planning} problem is
more specific, it consists in identifying paths that do not
intersect any obstacle.

In this project we are interested in generating smooth
paths. Smoothness is a desirable property that is
frequently presented in literature for the planar case. Consider, for
instance, the papers \cite{maekawa}, \cite{ho-liu}, \cite{li} and
\cite{giannelli}. The first considers an interesting combined
approach to the problem: analytical to find a smooth curve, and
stochastic to locate a desired cusp on it. On the second, they
concentrate in finding a curvature bounded path starting from a
\acf{VD} constructed accordingly to the environment. The third work
focuses on
the process of transforming an existing polyline path in a smooth
curve. In the last, they used
\acf{PH} curves that have interesting features for the \acf{CAM} field
\cite{farouki}.

In regards of spatial path planning, the smoothing problem is less
covered in literature. For instance, in \cite{hrabar} it is not clear
how a smooth path is obtained from the initial polynomial chain. In
\cite{yang} the smoothness is considered, but the method is not
optimal because it consists in alternating smoothing and
obstacle-checking phases until an admissible solution is
obtained. Other works, like \cite{aghababa} and \cite{kroumov}, use 
stochastic methods to achieve smoothness.

\bs curves are a reference standard in \acf{CAD} and \acf{CAGD}
\cite{hughes}\cite{foley}\cite{farin}\cite{farin2}. Thus, we decide to
develop a 3D path planning application using this kind of curves, as
\cite{yang} do. However, as earlier mentioned,
\cite{yang} uses a \emph{try and check} approach to the curve
smoothing, we present a method that finds a smooth
curve on the first 
attempt instead.

The considered topic is highly interdisciplinary. In fact we integrate
in this project an extended set of
competencies acquired during the courses. We apply notions of
\emph{linear algebra} for the collision checks; \emph{numerical
  analysis} for the curves design; \emph{computational
  geometry},
\emph{graph theory}, \emph{probability} and \emph{algorithm theory}
for the algorithms design; and, finally, \emph{theoretical computer
  science} for cost analysis.

We focus on finding a trade-off between having a short curve, a
smooth curve, and keeping the time complexity low.
Different solutions are explored, with different qualitative effects
on the curve.

Regarding the scene representation, different kinds of
polyhedral obstacle are considered.

A framework in Python is developed using \acf{VTK} for the graphic
output. We use a roadmap method based on \acfp{VD} to create a
graph (details in
\cref{sec:baseGraph}) that is the base
structure for the project. Using such structure, three
different solutions are presented.
\begin{enumerate}
\item The first method benefits from the \acf{CHP} of \bs
  curves (\cref{sec:convexHull}). A transformation is applied on the
  graph such that every path in it can be used as a control polygon
  for an obstacle-free curve (\cref{sec:trigraph}). Therefore, the
  algorithm selects the shortest path in the transformed graph and
  builds the curve on it (\cref{sec:inter1}).
\item The second method still benefits from the \ac{CHP},
  but it picks the shortest path directly in the base graph. If
  violations of the \ac{CHP} emerge in it, then rectification
  measures are taken (\cref{sec:inter2}).
\item The third method uses a probabilistic approach. Starting from
  the shortest path in the original graph, it performs a simulated
  annealing optimization (\cref{sec:simulatedAnnealing}) that converges
  in a state where we have an optimal trade-off between having a short curve,
  and low curvature and torsion peaks (\cref{sec:inter3}).
\end{enumerate}

This document consists of three parts. The first
(\cref{prt:stateOfArt})
is dedicated to the state of the art: we provide a survey of
different topics and algorithms related to \emph{motion planning}.

The
second part (\cref{prt:project}) is committed to describing
all the different parts of the algorithm. In detail:
\begin{itemize}
\item \cref{cha:prerequisites} gives to the reader all the necessary notions
  to understand the rest of the chapter;
\item \cref{cha:scene} describes how the
  environment and the resulting curve are represented;
\item Finally in
\cref{cha:algorithm} we describe how to obtain the basic structures
(\cref{sec:polChain}), how to avoid the obstacles using the three
methods described before (\cref{sec:obsAvoid}), and how to improve the
obtained curve simplifying the control polygon (\cref{sec:postPro}),
increasing the curve degree (\cref{sec:degreeInc}) and changing the
\bs knot vector (\cref{sec:knotSel}).
\end{itemize}

The third part (\cref{prt:evaluation}) describes the instruments used
to implement the algorithms (\cref{cha:codeStructure}) and presents
a series of tests with
different scenes and configurations (\cref{cha:testing}) with their
conclusions (\cref{cha:conclusions}).

In conclusion, \cref{cha:appendix} contains all the source code of the
application.

\end{document}

%%% Local Variables:
%%% mode: latex
%%% TeX-master: "../dissertation"
%%% End:
