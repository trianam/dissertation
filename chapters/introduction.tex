\documentclass[dissertation.tex]{subfiles}
\begin{document}
\chapter{Introduction}
The design of \emph{motion planning} strategies plays a fundamental
role in different applications, from robotics to scientific
visualization. \emph{Path planning} problem is
narrower, it includes the identification of paths that do not
intersect any obstacle.

In this project we are interested in generating smooth paths that are
present literature for the planar case \cite{giannelli}.


 Such paths are designed using
\bs curves, a reference standard in \ac{CAD} and \ac{CAGD}
\cite{hughes}\cite{foley}\cite{farin}\cite{farin2}.

The considered topic is highly cross-disciplinary. In fact we designed
this project with the help of an extended set of
competencies acquired following the courses. We applied notions of
\emph{linear algebra} for the collision checks; \emph{numerical
  analysis} for the design of the curves; \emph{computational
  geometry},
\emph{graph theory}, \emph{probability} and \emph{algorithm theory}
for the design of the 
algorithms; and finally \emph{theoretical computer science} for the
analysis of the costs.

We focused on finding a trade-off between having a short curve, a
smooth curve, and keeping the time complexity low. We explored
different solutions with different collocations on the trade-off
triangle and with different qualitative effects on the curve.

We developed a framework in Python using \ac{VTK} for the graphic
output. We used a roadmap method based on \acp{VD} for creating a
graph (details in
\cref{sec:baseGraph}), that is the base
structure for the project. Using such structure we developed three
different solutions.
\begin{enumerate}
\item The first method benefits from the \acf{CHP} of \bs
  curves (\cref{sec:convexHull}). it applies a transformation on the
  graph such that every path in it can be used as a control polygon
  for an obstacle-free curve (\cref{sec:trigraph}). Therefore the
  algorithm selects the shortest path on the transformed graph and
  build the curve on it (\cref{sec:inter1}).
\item The second method still benefits from the \ac{CHP},
  but it picks the shortest path directly on the base graph. If
  violations of the \ac{CHP} emerge on it, then rectification
  measures are taken (cref{sec:inter2}).
\item The third method uses a probabilistic approach. Starting from
  the shortest path in the original graph, it performs a simulated
  annealing optimization (\cref{sec:simulatedAnnealing}) that converge
  in a state were the desired trade-off between having a short curve,
  and low curvature
  and torsion peaks, is optimal (\cref{sec:inter3}).
\end{enumerate}

This document is structured as follows, we have three parts. The first
(\cref{prt:stateOfArt})
Is dedicated to the state of the art, we provide an survey of
different topics and algorithms related to \emph{motion planning}.

The
second part (\cref{prt:project}) is committed to describing
all the different parts of the algorithm. In detail
\cref{cha:prerequisites} gives to the reader all the necessary notions
to understand the following parts. \cref{cha:scene} describes how the
environment and the resulting curve are represented. Finally in
\cref{cha:algorithm} we describe how to obtain the basic structures
(\cref{sec:polChain}), how to avoid the obstacles using the three
methods described before (\cref{sec:obsAvoid}), and how to improve the
obtained curve simplifying the control polygon (\cref{sec:postPro}),
increasing the curve degree (\cref{sec:degreeInc}) and changing the
\bs knot vector (\cref{sec:knotSel}).

The third part (\cref{prt:evaluation}) describe the instruments used
for implementing the algorithms (cref{cha:codeStructure}) and presents
a series of tests with
different scenes and configurations (\cref{cha:testing}). Furthermore
we deduce conclusions from the execution of the tests
(\cref{cha:conclusions}).

\end{document}

%%% Local Variables:
%%% mode: latex
%%% TeX-master: "../dissertation"
%%% End:
