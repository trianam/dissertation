\documentclass[dissertation.tex]{subfiles}
\begin{document}
\chapter{Conclusions}\label{cha:conclusions}
In this chapter we describe the evidences that emerge from the tests,
furthermore we discuss possible future improvements of the project.

\section{Tests analysis}
We present three scenes for doing the tests. Scene \sceneA consists
in 10 obstacles randomly disposed, scene \sceneAb is the same scene
with a more dense graph, scene \sceneB have 100 obstacle and \sceneC has
only one bucket-shaped obstacle.

We set the starting and ending points
to be at the extremes of the bounding box for scene \sceneA and scene
\sceneAb - 
i.e. the objective of the tests in such scenes is to cross the area
with the obstacles. For scene \sceneB we set the starting point in the
centre of the crowded area, and the objective is to manage to exit
from the area. For scene \sceneC we set the starting point inside the bucket
and we want to arrive under it.

Starting with the test for methods \metA and \metB, we manage to test
all the possible configuration of scenes, degree, method, post
processing and knot partition. For method \metC we tested 5 different
parameter sets for the \ac{SA}.

Regarding the performances, the fastest method is \metB, for having an
idea of the temporal scales consider that, on a quad core Intel
i5-2430M CPU at 2.40GHz with 8 Gb of RAM, an execution in scene
\sceneA, with post processing and adaptive knot partitions, takes:
\begin{itemize}
\item 76 seconds for method \metA;
\item 11 seconds for method \metB.
\end{itemize}
It is difficult to compare method \metC because of the different
parameter sets, anyway an execution of it in scene \sceneA, with
configuration \annA, takes 168 seconds.

First of all we notice that the application of the adaptive partition
results in a deterioration of the curvature plots - i.e. an increase
of the curvature peaks - for the experiments with scenes \sceneA,
\sceneAb and \sceneB. Instead it results in an improvement with
experiments with \sceneC.

The post processing fulfills the
objective of simplify the path. Consider for instance test 33
(\cref{fig:test33}) where the curvature plot has different
peaks. After the application of post processing we obtain test 34
(\cref{fig:test34}) where the curvature peaks are mitigated.

Regarding the degree increase algorithm, it is working for the
curvature: the plots are continue for degree 3 and continue and
smooth for degree 4. Unfortunately it is not working for the torsion
(not shown on the tests) because by adding aligned vertices we force
plane changes on zero-curvature points, where the torsion is not
defined.

Solution three produces high quality curves (see tests from 74 to 99),
with low peaks of curvature and torsion. Furthermore this solution do
not suffer from the problem with degree increase mentioned before, and
the plots of the torsion are good.

An important drawback of this
solution is that the collision check is made in a discrete way, thus,
depending on the parameters, it is possible that the path intersects
an obstacle without noticing it.
Other drawbacks are the slower
execution time and the difficulty in finding the right values of the
annealing parameters. In fact wrong values of warming, or insufficient
trials can \emph{freeze} the system in a not optimal
status. Furthermore it is necessary to adapt the parameters to
different problems. For instance the configuration \annB and \annC are
not fitted for tests from 94 to 99, using those settings is not enough
to let the system converge in an admissible state.



\section{Future improvements}
Considering the different benefits and drawback of the implemented
solutions, we advice to insists in mixed approaches to the
problem, analytical and stochastic.

An interesting new possible solution can be implementing a
stochastic optimization on the path obtained from solution 1 or
solution 2, that is obstacle-free guaranteed. Such hypothetical stochastic
optimization must avoid states that violates the \ac{CHP}, and it can
work directly on the state space, without the Lagrangian
relaxation. In fact in that scenario the initial status is already
obstacle-free, furthermore we believe that the optimization process do
not need to explore too much the state space trespassing obstacle
zones.

We believe that the described process can be very effective in
improving curvature, torsion and length of the path. It can obtain
curves with the quality of the implemented third solution, without the
drawbacks of it: the slow computation and the possible collision
errors due to the discretization of the inclusion checks.

\end{document}

%%% Local Variables:
%%% mode: latex
%%% TeX-master: "../dissertation"
%%% End:
