\documentclass[dissertation.tex]{subfiles}
\begin{document}
\chapter{Algorithm}
In this chapter we analyze step-by-step the algorithms that implements
the different parts of the program. We do this with the help of the
test scene in \cref{fig:empty}.
\image{scrEmpty.png}{Initial scene.}{fig:empty}

The general idea is to use a B-spline of a certain degree that use as
control polygon a polygonal chain extracted in a graph
derived from a Voronoi diagram method and is clamped to the extremes
of such polygon. In \cref{sec:polChain} we explain in detail how to
build such polygonal chain. Before using the chain as a control
polygon for the B-spline, it is refined and adjusted - as explained in
detail in \cref{sec:obsAvoid} and \cref{sec:degreeInc} - such that is
possible to apply a B-spline with the required degree that don't hit
any obstacle. We implemented a method for an optional adaptive
arrangment of the inner nodes of the B-spline partition described in
\cref{sec:nodeSel} and an optional post processing for the
simplification of the path described in \cref{sec:postPro}.
\section{Polygonal chain}\label{sec:polChain}
In the first phase the objective is to extract a polygonal chain from
the scene, such that the extremes coincides with the start point $s$
and the end point $e$. For doing this we calculate the shortest path
in a graph that is constructed using an adaptation to three
dimensions of a well known bidimensional method (see
\cite{bhattacharya}, \cite{ho-liu} and \cite{seda-pich}) that use
Voronoi diagrams as base.

\subsection{Base structure}
First we
start distributing points on the faces of the obstacles and on
an invisible bounding box, as in \cref{fig:sites}.
\image{scrSites.png}{Scene with Voronoi sites.}{fig:sites}
The sites are distributes using a recursive method, for each triangle
of the scene\footnote{The scene is composed exclusively of triangles,
  see \cref{cha:scene} for details.} we add three points - one for each
vertex, if not already added before - and then we calculate the area
of the triangle, if the area is bigger than a threshold we decompose
the triangle in four triangles adding three more vertexes on the
midpoints of the edges of the original triangle as in
\cref{fig:triangleDec}, and repeating the process recursively for
each new triangle.
\begin{myfig}{Decomposition of a triangle.}{fig:triangleDec}
  \begin{tikzpicture}[scale=2]
    \coordinate (a1) at (1,0);
    \coordinate (b1) at (2,1);
    \coordinate (c1) at (3,0.3);

    \path[obstacle] (a1) -- (b1) -- (c1) -- (a1);

    \coordinate (dist) at (3.5,0);
    \coordinate (a2) at ($ (a1) + (dist) $);
    \coordinate (b2) at ($ (b1) + (dist) $);
    \coordinate (c2) at ($ (c1) + (dist) $);

    \path[obstacle] (a2) -- (b2) -- (c2) -- (a2);
    \foreach \p in {a1,b1,c1,a2,b2,c2}
    \filldraw[site] (\p) circle (2pt); 

    \coordinate (ab) at ($ (a2)!0.5!(b2) $);
    \coordinate (bc) at ($ (b2)!0.5!(c2) $);
    \coordinate (ac) at ($ (a2)!0.5!(c2) $);

    \path[obstacleTract] (ab) -- (bc) -- (ac) -- (ab);

    \foreach \p in {ab,bc,ac}
    \filldraw[siteHigh] (\p) circle (2pt); 
  \end{tikzpicture}
\end{myfig}

We construct the Voronoi diagram using the Fortune's algorithm on
those points as input sites, and we build a graph
$$G=(V,E)$$
using the vertexes
of the Voronoi cells as graph nodes $V$, and the edges of the cells\footnote{Ignoring the
  infinite edges.} as graph edges $E$. After we prune such graph deleting every edge that
intersect an obstacle face, using the methods explained in \cref{sec:intersections}.
\image{scrGraph.png}{Scene with pruned graph.}{fig:graph}

The result, visible in \cref{fig:graph}, is a graph that embrace the
obstacles like a cobweb where the possible paths on it are roughly
equidistant between the obstacles.

In the bidimensional case the equivalent method implies distributing
the sites in the edges of the polygonal obstacles and then pruning the
graph when an edge of the graph intersect an edge of the obstacle as
in \cref{fig:voronoi2d}.
\image{voronoi2d.png}{Voronoi graph in 2D before pruning (a) and after
  pruning (b).}{fig:voronoi2d}
We decided to extend the method in 3 dimensions distributing points in
the whole faces of obstacles, an alternative would be distributing
points only along the edges of the obstacles.

On the graph obtained $G$ we attach the desired start and end
points, and we can obtain a path between the two points using an
algorithm like Dijkstra. Before using that path as a control polygon
we need to take into account the degree of the B-spline and the
position of the obstacles, the details are in \cref{sec:obsAvoid} and
\cref{sec:degreeInc}.

\subsubsection{Complexity considerations}
Fortune's algorithm runs in time $\bigO(|Is|\log |Is|)$ where $Is$ is
the set 
of input sites. If we impose a maximum area for the obstacles -
i.e. the area of an obstacle cannot be a function of the number of
the obstacles - then $|Is|=\bigO(|Ob|)$ where $Ob$ is the set of
obstacles.

In conclusion the time cost for the creation of the graph
is
\begin{equation}
  \label{eq:costGraph}
  \bigO(|Ob|\log |Ob|)
\end{equation}
and the number of the vertexes in the graph
is 
\begin{equation}
  \label{eq:numV}
  |V|=\bigO(|Is|)=\bigO(|Ob|)
\end{equation}
because the number of vertexes in the resulting graph is the
same order of number of input sites.

If we make the hypothesis of having maximum degree $k$ in $G$ -
we have maximum $k$ neighbours for each vertex in $V$ - then we have
that
\begin{equation}
  \label{eq:numE}
  |E|=\bigO(k|V|)=\bigO(k|Ob|).
\end{equation}
In the worst case $k=|V|$ and $|E|=\bigO(|V|^2)$ but for Voronoi
diagrams in plane there is a property that if we have $n$ input sites
that lays in a circumference, without any other site inside the
circumference, then the center of the circumference is a vertex shared
by $n$ cells. The same property is extensible in space with
spheres. We can make the assumption that no more than three sites can lays
in a circumference so no vertex can have more than three neighbours,
or the same with four vertexes in sphere, and this assumption is
plausible because we use floating point numbers for the coordinates of
the vertexes of the obstacles, and is quite improbable that more than
four points lays in a sphere. With this assumption $k$ is a constant,
and \cref{eq:numE} become
\begin{equation*}
  |E|=\bigO(|V|)=\bigO(|Ob|).
\end{equation*}

For pruning the graph of every edge that intersect obstacles we need
to solve a system of four unknowns in four equations for every edge
and every obstacle's face, so we have a cost of
\begin{equation}
  \label{eq:costPruning}
  \bigO(|E|\cdot|Ob|)=\bigO(k|Ob|^2)
\end{equation}
and if we make the assumption of $k$ constant
\begin{equation*}
  \bigO(|Ob|^2)
\end{equation*}

\subsection{Triple's graph}\label{sec:trigraph}
Before calculating the shortest path with Dijkstra, we
transform the graph of vertexes in a graph containing all the triples
of three adjacent vertexes in the original graph. This because we want
to filter the triples for collisions as described in
\cref{sec:inter1}. We call the transformed graph
$$G_t=(V_t,E_t)$$

The original graph $G$ is not directed and weighted
with the distance from vertex to vertex. The transformed graph $G_t$ is
directed and weighted, where if in $G$ the nodes $A$
and $B$ are
neighbouring, and the nodes $B$ and $C$ are neighbouring, then $G_t$
have the two nodes $(A,B,C)$ and $(C,B,A)$. In
$G_t$ a node $(A_1,B_1,C_1)$ is a predecessor of
$(A_2,B_2,C_2)$ if $B_1=A_2$ and $C_1=B_2$, and the weight of the arc
from $(A_1,B_1,C_1)$ to $(A_2,B_2,C_2)$ in $G_t$ is
equal to
the weight of the arc from $A_1$ to $B_1(=A_2)$ in $G$.

The transformation of the graph is useful only for the obstacle
avoidance algorithm of
\cref{sec:inter1}, theoretically is possible to bypass such
transformation for the algorithm on \cref{sec:inter2}.

\subsubsection{Complexity considerations}
If we suppose a maximum degree $k$ for each vertex in the graph $G$ -
i.e. each vertex in $V$ can have maximum $k$ edges attached to
it - then the number of triples in the transformed graph $G_t$ is 
\begin{equation}
  \label{eq:numTriples}
  |V_t|=|V|\cdot k\cdot(k-1)=\bigO(k^2|V|)
\end{equation}
because for each vertex $v$ in $G$ we need to
consider all the neighbours of $v$ and the neighbours of the neighbours
of $v$ (excluded $v$).

For how we defined the triples neighbour rule in $G_t$ we have
that each triple is a predecessor of maximum $k-1$ other triples, for
instance $(A,B,C)$ in $V_t$ is the predecessor of all the triples
$(B,C,*)$ where $*$ can be one of the $k$ neighbours
of $C$ in $V$ excluded $B$. So the number of edges in $G_t$ is
\begin{equation}
  \label{eq:numEdgesTriples}
  |E_t|=|V_t|\cdot (k-1)=\bigO(k|V_t|)=\bigO(k^3|V|)
\end{equation}

Also the time cost for the creation of $G_t$ is
\begin{equation}
  \label{eq:costVt}
  \bigO(k^2|V|)=\bigO(k^2|Ob|)
\end{equation}
because the algorithm for creating the transformed graph work in a
straightforward method, building each triple and connecting it one by
one.

\section{Obstacle avoidance}\label{sec:obsAvoid}
Before using the polynomial chain extracted using the algorithm
explained in \cref{sec:polChain} as a control polygon for the
B-spline, we need to deal with a
problem: every possible path in the graph $G$ is
free from collisions by construction - in fact we prune the graph of
every edge that intersect an obstacle - but this don't guarantees that
the curve will be free from obstacles. This concept is expressed in \cref{fig:intersect}.
\begin{myfig}{Schematic of B-spline that intersect an obstacle in the plane.}{fig:intersect}
  \begin{tikzpicture}
    \path[obstacle] (1,0) -- (2,1) -- (3,0) -- (1,0);
    \draw[controlPoly] (0,0) -- (2,2) -- (4,0);
    \draw[spline] (0,0) to [bend left=40] (4,0);

    \filldraw[controlVert] (0,0) circle (2pt);
    \filldraw[controlVert] (2,2) circle (2pt);
    \filldraw[controlVert] (4,0) circle (2pt);
  \end{tikzpicture}
\end{myfig}

In this chapter we make the hypothesis of using
quadratic B-splines\footnote{Degree 2, order 3.}, in
\cref{sec:degreeInc} is explained how is possible to use curves with
higher degree. With this assumption we can exploit the convex hull
property explained in \cref{sec:bsplineProp} and assert that the
resulting curve is contained inside the union of all the triangles
composed of three consecutive control vertexes of the control
polygon. Using that property we can resolve the problem of the
collision maintaining all the triangles in the control polygon free
from collision with the triangles that compose the faces of the
obstacles. Note that the property of convex hull of quadratic
B-splines is valid also in space, so the convex hull is still composed
of planar figures - specifically of triangles - like the faces of the
obstacles. This simplify all the check for the collisions because they
are all between triangles in space and we can use the methods
described in \cref{sec:intersections}.

We designed two different algorithms for dealing with the collision
problem. The first solution described in \cref{sec:inter1} implements
a modified version of Dijkstra's
algorithm that find the shortest path from start to end in the graph
such that all the triangles formed of consecutive points in such path
are free from collisions. The second solution described in
\cref{sec:inter2} use the classical Dijkstra's algorithm for finding
the shortest path from start to end in the graph $G$, checking later for
collisions in the triangles formed of consecutive points in such path
and when one is found some actions are taken for dealing with that.

\subsection{First solution: Dijkstra's algorithm in triples}\label{sec:inter1}
The first solution of the problem exploit the graph $G_t$ obtained as
explained in \cref{sec:trigraph}. Before applying Dijkstra's algorithm
to $G_t$ all the triples are filtered checking if the
triangle composed of the vertexes of a triple intersect an obstacle's
face. If a triple intersect an obstacle then is removed from the graph
and so a path cannot pass from such vertexes in that order.

Note that if a triple $(A,B,C)$ is removed from $V_t$ - and
consequently also the triple $(C,B,A)$ - not necessarily this exclude
the three vertexes $A$, $B$, $C$ from being part of the final
polynomial chain. For instance\footnote{In the plane, this graph cannot be
  obtained using the procedure based on Voronoi diagrams explained in
  \cref{sec:polChain}, but a similar situation is plausible
  considering Voronoi cells in space.} in
\cref{fig:exampleTriples} the triple
$(A,B,C)$ and $(C,B,A)$ will be removed from $G_t$ because
the corresponding triangle intersect the obstacle, and the path
$D\rightarrow A\rightarrow B\rightarrow C\rightarrow E$ cannot be an
admissible path. This don't preclude the nodes $A$, $B$ and $C$ to be part
of the final admissible path $D\rightarrow A\rightarrow B\rightarrow E\rightarrow C\rightarrow E$.
\begin{myfig}{Example of triples.}{fig:exampleTriples}
  \begin{tikzpicture}
    \coordinate (D) at (-1,1);
    \coordinate (A) at (0,0);
    \coordinate (B) at (2,2);
    \coordinate (C) at (4,0);
    \coordinate (E) at (3,2);
    \coordinate (F) at (5,1);
    \path[obstacle] (1,-0.5) -- (2,1) -- (3,-0.5) -- (1,-0.5);
    \draw[controlPoly] (D) -- (A) -- (B) -- (C) -- (F);
    \draw[controlPoly] (B) -- (E) -- (C);

    \filldraw[controlVert] (D) circle (2pt);
    \filldraw[controlVert] (A) circle (2pt);
    \filldraw[controlVert] (B) circle (2pt);
    \filldraw[controlVert] (C) circle (2pt);
    \filldraw[controlVert] (E) circle (2pt);
    \filldraw[controlVert] (F) circle (2pt);

    \node[above=0.5em] at (D) {$D$};
    \node[below=0.5em] at (A) {$A$};
    \node[above=0.5em] at (B) {$B$};
    \node[below=0.5em] at (C) {$C$};
    \node[above=0.5em] at (E) {$E$};
    \node[above=0.5em] at (F) {$F$};
  \end{tikzpicture}
\end{myfig}

On the cleaned transformed graph is possible to find the shortest path on the
triples
$$
(A_0,B_0,C_0), (A_1,B_1,C_1),\dots,(A_i,B_i,C_i),\dots,(A_n,B_n,C_n)
$$
 using
an algorithm like Dijkstra, and then extracting the shortest
path on the vertexes taking the central vertex $B_i$ of every
triple $(A_i,B_i,C_i)$ of the path plus the extremes $A_0$ and $C_n$
of the first and last triples, obtaining
$$
A_0,B_0,B_1,\dots,B_i,\dots,B_{n-1},B_n,C_n.
$$

\imageDouble{scrSolution1a.png}{scrSolution1b.png}{Effects of application of solution one.}{fig:sol11}
\imageDouble{scrSolution1a2.png}{scrSolution1b2.png}{Effects of
  application of solution one, other view point.}{fig:sol12}
In \cref{fig:sol11} and \cref{fig:sol12} is shown the effect of the
application of the first solution. The triangle formed by the second,
the third and the
fourth vertexes from the right on the left figure of \cref{fig:sol11}
is colliding with the obstacle in the back, on the right figure there
is the path obtained applying the solution, where no triangles in the
path collides with obstacles. On \cref{fig:sol12} is visible another
point of view of the three vertexes that collides.

\subsubsection{Complexity considerations}
For each triple and each obstacle we need to solve a $4\times 4$
linear system for the
collision check, so in total the cost is
\begin{equation*}
  \bigO(|V_t|\cdot |Ob|)
\end{equation*}
and for \cref{eq:numV} and \cref{eq:numTriples} this is equal to
\begin{equation}
  \label{eq:costColl}
  \bigO(|Ob|^2 k^2).
\end{equation}

The cost of applying Dijkstra's algorithm\footnote{In the worst case
  where no triples are removed in the cleaning phase.} in $G_t$ is
\begin{equation}
  \label{eq:costDijkstraTriples}
  \bigO(|E_t|+|V_t|\log |V_t|) = \bigO(k^3|V|+k^2|V|\log(k^2|V|)
\end{equation}
This cost have two special cases, if $G$ is a clique - i.e. each
node in $V$ is connected to every other node - then $k=|V|-1$ and the
cost is
\begin{equation*}
  \bigO(|V|^4);
\end{equation*}
if $k$ is constant - i.e. don't grow with the growing of $|V|$ - the
cost is
\begin{equation*}
  \bigO(|V|\log|V|).
\end{equation*}
The latter case is the more plausible if we assume the hypothesis that
no more than four input sites in space (three in plane) can be on the
same sphere in space (circumference in plane), in fact in that case
every Voronoi cell cannot have a vertex with more than four edges
connected in space (three in plane).

If we sum all the costs \cref{eq:costGraph}, \cref{eq:costPruning},
\cref{eq:costVt},
\cref{eq:costColl} and
\cref{eq:costDijkstraTriples} we obtain:
\begin{equation*}
  \bigO(k^2|Ob|^2+k^3|Ob|)
\end{equation*}
where all the other factors are absorbed in those two. If we have $k$
constant as we said before then we have an overall cost of
\begin{equation*}
  \bigO(|Ob|^2)
\end{equation*}
That derive from the collision check control.

We can improve this result if we divide the algorithm in two parts, we
can construct the graph only once with the heavy cost $\bigO(|Ob|^2)$
and then using
it in different trials with different starting and ending points with
a reduced cost of
\begin{equation*}
  \bigO(|Ob|\log |Ob|)
\end{equation*}
for the routing.

\subsection{Second solution: clean path}\label{sec:inter2}
The First solution is interesting from an algorithmic point of view,
but is not very practical because it throws all the triples that
intersect an obstacle. We tried another
solution that use another approach, the second solution use the
shortest path obtained
from the Voronoi's graph $G$ directly using Dijkstra's algorithm,
without removing any triple. On that path
- that we call $P$ in this section - we
check every triple of consecutive vertexes, and if it collides with an
obstacle's face then we take countermeasures. For instance if the path
is composed from the vertexes:
\begin{equation*}
  P=V_0,V_1,\dots,V_n
\end{equation*}
then we check if every one of the triangles 
\begin{eqnarray*}
T_0 &=& \triangle V_0V_1V_2\\
T_1 &=& \triangle V_1V_2V_3\\
&\cdots&\\
T_i &=& \triangle V_iV_{i+1}V_{i+2}\\
&\cdots&\\
T_{n-3} &=& \triangle V_{n-3}V_{n-2}V_{n-1}\\
T_{n-2} &=& \triangle V_{n-2}V_{n-1}V_n
\end{eqnarray*}
intersect one of the triangles that compose the faces of the
obstacles, where $\triangle V_iV_jV_k$ denotes the triangle with
vertexes the points $V_i$, $V_j$ and $V_k$.

In \cref{sec:intersections} we explain how to measure if a triangle
collides with another triangle and that when a collision is found we
obtain also the barycentric coordinates of the point were one of the
segment of a triangle intersect the other triangle express in the
vertexes of the second one.

Consider also that $G$ is pruned from all the edges that intersect any
obstacle, so none of the edges of the triangles $T_i$ can intersect
an obstacle, and the only possibility is that a segment or more - two
segments for each obstacle face if we ignore special cases - of the
obstacles intersect a
triangle $T_i$. So for each $T_i$ we have a possibly empty set of point
of intersection with the edges of the obstacles - we call it $O$.
\begin{myfig}{$T_i$ and obstacles
    represented by the points of intersection.}{fig:triangleIntersection}
  \begin{tikzpicture}
    \coordinate (P) at (-1,0);
    \coordinate (A) at (0,0);
    \coordinate (B) at (4,5);
    \coordinate (C) at (8,1);
    \coordinate (D) at (9,1);
    \coordinate (O1) at (barycentric cs:A=0.2,B=0.6,C=0.2);
    \coordinate (O2) at (barycentric cs:A=0.5,B=0.3,C=0.2);
    \coordinate (O3) at (barycentric cs:A=0.3,B=0.2,C=0.5);
    \coordinate (W1) at (barycentric cs:A=0.4,B=0.6,C=0.);
    \coordinate (W2) at (barycentric cs:A=0.,B=0.6,C=0.4);

    \draw[controlPoly] (A) -- (B) -- (C);
    \draw[controlPolyTract] (P) -- (A);
    \draw[controlPolyTract] (C) -- (D);
    \draw[controlPolyTractHigh] (W1) -- (W2);
    \draw[controlPolyTractHigh] (A) -- (C);

    \filldraw[controlVert] (A) circle (2pt);
    \filldraw[controlVert] (B) circle (2pt);
    \filldraw[controlVert] (C) circle (2pt);
    \filldraw[obstaclePoint] (O1) circle (2pt);
    \filldraw[obstaclePoint] (O2) circle (2pt);
    \filldraw[obstaclePoint] (O3) circle (2pt);
    \filldraw[controlVertHigh] (W1) circle (2pt);
    \filldraw[controlVertHigh] (W2) circle (2pt);

    \node[below=0.5em] at (A) {$V_i$};
    \node[above=0.5em] at (B) {$V_{i+1}$};
    \node[below=0.5em] at (C) {$V_{i+2}$};
    \node[below=0.2em] at (O1) {$O_1\equiv O_n$};
    \node[left=0.2em] at (O2) {$O_2$};
    \node[below=0.2em] at (O3) {$O_3$};
    \node[left=0.5em] at (W1) {$W_1$};
    \node[right=0.5em] at (W2) {$W_2$};
  \end{tikzpicture}
\end{myfig}

In \cref{fig:triangleIntersection} we have an example of the triangle
\begin{equation*}
  T_i = \triangle V_iV_{i+1}V_{i+2} 
\end{equation*}
that is
intersected by obstacles in the points
\begin{equation*}
  O = \{O_1,O_2,O_3\}.
\end{equation*}
where each one of the points in $O$ is expressed in barycentric
coordinates of the vertexes $V_i$, $V_{i+1}$ and $V_{i+2}$ of the
triangle:
\begin{eqnarray*}
  O_1&=&\alpha_1 V_i+\beta_1 V_{i+1}+\gamma_1 V_{i+2}\\
  O_2&=&\alpha_2 V_i+\beta_2 V_{i+1}+\gamma_2 V_{i+2}\\
  O_3&=&\alpha_3 V_i+\beta_3 V_{i+1}+\gamma_3 V_{i+2}
\end{eqnarray*}
where $\alpha_i+\beta_i+\gamma_i=1$ for $i=1,2,3$.

We want to deal with the collision adding vertexes in the control
polygon such that consecutive triangles are free from obstacles. We
obtain this adding a control vertex $W_1$ between $V_i$ and $V_{i+1}$
and $W_2$ between $V_{i+1}$ and $V_{i+2}$. Also we add those two
points in a manner such that the segment $\overline{W_1W_2}$ is
parallel to the segment $\overline{V_iV_{i+2}}$ and such that
$\overline{W_1W_2}$ pass through the obstacle point $O_n$ nearer to
$V_{i+1}$ ($O_1$ in \cref{fig:triangleIntersection}) keeping the
triangle $\triangle W_1V_{i+1}W_2$ - and also
the degenerated triangles $\triangle V_iW_1V_{i+1}$ and $\triangle
V_{i+1}W_2V_{i+2}$ - free from obstacles.

When we check for collisions between a segment and a triangle we
resolve a system of four unknowns in
four equations and the solutions are the barycentric coordinates of
the point of collision, when we have all the coordinates of the points
in $O$ we can obtain $O_n$ picking the one with the biggest $\beta$
and then, using the corresponding $\beta_n$, we can obtain
\begin{eqnarray*}
  W_1&=&\beta_n V_{i+1}+(1-\beta_n)V_i\\
  W_2&=&\beta_n V_{i+1}+(1-\beta_n)V_{i+2}.
\end{eqnarray*}

\imageDouble{scrSolution2a.png}{scrSolution2b.png}{Effects of application of solution two.}{fig:sol21}
\imageDouble{scrSolution2a2.png}{scrSolution2b2.png}{Effects of
  application of solution two, other view point.}{fig:sol22}
In \cref{fig:sol21} and \cref{fig:sol22} we can see the effects of the
application of the
solution two to a tract of the curve. The original tract of control
polygon is on the left image and, on \cref{fig:sol22}, is composed of
the lower
visible vertex corresponding to $V_i$, the
upper visible vertex corresponding to $V_{i+2}$, and the one in
the middle on the left corresponding to $V_{i+1}$. The same vertexes
in the other \cref{fig:sol21} are the second, the third and the fourth
from the right. The triangle
composed of those vertexes
collide with the obstacle on the right and the two new vertexes $W_1$
and $W_2$ are the second visible from below and the second visible
from above on the right figure.

\subsubsection{Complexity considerations}
For this solution we have still the cost of \cref{eq:costGraph} and
the cost of \cref{eq:costPruning} for
the creation and pruning of the graph $G$, then we need to apply Dijkstra's
algorithm in $G$ for obtaining $P$ with a cost
\begin{equation*}
  \bigO(|E|+|V|\log |V|).
\end{equation*}
For \cref{eq:numV} \cref{eq:numE} that cost is equal to
\begin{equation*}
  \bigO(k|Ob|+|Ob|\log |Ob|)
\end{equation*}
and if we make the assumption of $k$ constant we have a cost
\begin{equation*}
  \bigO(|Ob|\log |Ob|).
\end{equation*}

For checking and removing the collisions in the path we need to
consider every face of obstacle in $Ob$ for every
triangle in $P$ and\footnote{if the
  number of faces of the obstacles don't grow with the number of
  obstacles.} the cost for doing this is
\begin{equation*}
  \bigO(|P|\cdot|Ob|).
\end{equation*}
where in the worst case $|P|=\bigO(|V|)=\bigO(|Ob|)$ and the cost
became
\begin{equation*}
  \bigO(|Ob|^2)
\end{equation*}

Summing up all the costs we have 
\begin{equation*}
  \bigO(|Ob|\log|Ob|+k|Ob|^2)
\end{equation*}
and if we consider $k$ constant
\begin{equation*}
  \bigO(|Ob|^2).
\end{equation*}

The cost is comparable with the one of the first solution, and also in
this case we can divide the algorithm in two parts constructing only
once $G$ with cost $\bigO(|Ob|^2)$ and using it for different
trials with a cost $\bigO(|P|\cdot|Ob|)$.

\subsection{Third solution: simulated annealing}\label{sec:inter3}
The others solutions of \cref{sec:inter1} and \cref{sec:inter2} have two
problem in common: both reject configurations in a prudent way
considering only the control polygon, and both don't optimize neither
length nor other parameters. On this section we describe a third
approach that we considered on the project, an approach that diverge
from the topic of numerical analysis.

We can see the problem of finding the shortest path as a constrained
optimization problem where a certain configuration of the control
vertexes (and
consequently the B-spline) is the state of the system, and we want to
minimize both the length of the control polygon (and consequently the
B-spline) and the peak in curvature and torsion of the B-spline under
the constraint that the B-spline must not intersect the obstacles. We
are interested in optimizing both length and curvature and torsion
peaks because we want a path that is short but also a path that is
smooth.

\subsubsection{Lagrangian relaxation applied to the project}
We can apply the concept explained in \cref{sec:lagrangianRelaxation}
to the project.

In the project the variable space $X$ is composed of all possible
configuration of the path, or in other words is the vector
$\mathbf{v}=(\mathbf{v}_1,\dots,\mathbf{v}_n)$ of all $n$ ordered
vertexes $\mathbf{v}_i=(x_i,y_i)$ of the
path. The \cref{eq:opt} is formulated in two different
versions in the project.
\begin{itemize}
\item The first one is:
  \begin{equation*}
    \begin{aligned}
      & \underset{\mathbf{v}}{\text{minimize}}
      & & length(\mathbf{v}) \\
      & \text{subject to}
      & & \left|bspline(\mathbf{v})\cap \bigcup_{i\in I}obstacle_i\right| = 0.
    \end{aligned}
  \end{equation*}
  Where $length(\mathbf{v})$ is the sum of the length of every edge of the
  path, that is the sum of the distances of every consecutive pair of
  vertexes $\mathbf{v_{i-1}}$ and $\mathbf{v_i}$; $bspline(\mathbf{v})$
  is the set of points of the \emph{B-Spline} obtained as explained in
  \cref{sec:spline} and \cref{sec:bsplines}, using the path as
  control polygon;
  $obstacle_i$ is the surface of the $i^{th}$ of $m$ obstacles, and
  $I=\{1,\dots,m\}$.
\item The second version is similar, but the function to minimize is
  $angle(\mathbf{v})$ instead of $length(\mathbf{v})$, where
  \begin{equation*}
    angle(\mathbf{v}) = \frac{\sum_{i=2}^{n-1}(1+\cos(\measuredangle \mathbf{v}_{i-1}\mathbf{v}_{i}\mathbf{v}_{i+1}))}{n-2},
  \end{equation*}
  and
  \begin{equation*}
    cos(\measuredangle
    \mathbf{v}_{i-1}\mathbf{v}_{i}\mathbf{v}_{i+1})=\frac{(\mathbf{v}_{i-1}-\mathbf{v}_{i})\cdot
      (\mathbf{v}_{i+1}-\mathbf{v}_{i})}{\|\mathbf{v}_{i-1}-\mathbf{v}_{i}\|\
      \|\mathbf{v}_{i+1}-\mathbf{v}_{i})\|}.
  \end{equation*}
\end{itemize}

The constraint function is not negative, and is calculated as the
ratio
\begin{equation*}
constraint(\mathbf{v}) = \frac{|\{\mathbf{p} \in spline(\mathbf{v})\ |\ \exists i\in\{1,\dots,m\}
  : \mathbf{p}\in obstacle_i \}|}{|\{\mathbf{p} \in spline(\mathbf{v})\}|}
\end{equation*}
where the points $\mathbf{p}$ of the spline are calculated in a discrete form. That
function is not negative, so the Lagrangian function correspondent to
\cref{eq:lagrangianFun} is
\begin{equation}\label{eq:lagrangianFunProj}
    L_d(\mathbf{v},\lambda)=length(\mathbf{v})+\lambda\cdot constraint(\mathbf{v}),
\end{equation}
or, alternatively, the correspondent one that use $angle(\mathbf{v})$ instead of $length(\mathbf{v})$.

\subsubsection{Annealing phase}
The target of the simulated annealing phase is to find the minimum
saddle points in
the curve represented by the
\cref{eq:lagrangianFunProj}.
\begin{algorithm}
 \caption{Annealing}\label{alg:annealing}
 \begin{algorithmic}[1]
   \Procedure{annealing}{$\mathbf{x}$}
   \State $\lambda\gets initialLambda$\label{alg:annealing:initialize}
   \State $T\gets initialTemperature$
   \While{not $terminationCondition()$}\label{alg:annealing:while}
   \ForAll{number of trials}\label{alg:annealing:for}
   \State $changeLambda\gets$ true with $changeLambdaProb$\label{alg:annealing:lambdaProb}
   \If{$changeLambda$}
   \State $\lambda'\gets neighbour(\lambda)$\label{alg:annealing:changeLambda}
   \State $\lambda\gets \lambda'$ with probability $\me^{-([energy(\mathbf{x},\lambda)-energy(\mathbf{x},\lambda')]^+/T)}$
   \Else
   \State $\mathbf{x}'\gets neighbour(\mathbf{x})$\label{alg:annealing:changeX}
   \State $\mathbf{x}\gets \mathbf{x}'$ with probability $\me^{-([energy(\mathbf{x}',\lambda)-energy(\mathbf{x},\lambda)]^+/T)}$
   \EndIf
   \EndFor
   \State $T\gets T\cdot warmingRatio$\label{alg:annealing:cooling}
   \EndWhile
   \EndProcedure
 \end{algorithmic}
\end{algorithm}

The \cref{alg:annealing} is the annealing process, on
\cref{alg:annealing:initialize} $\lambda$ and the
temperature are initialized; the \emph{while} on
\cref{alg:annealing:while} is the main loop and the terminating
condition\footnote{For the animation there isn't a terminating
  condition.} is given by a minimum temperature or a minimum variation of
energy between two iterations; the \emph{for} at
\cref{alg:annealing:for} repeat the annealing move for a certain
number of trials, on each iteration the algorithm probabilistically
try to make a move of the state of the system, first on
\cref{alg:annealing:lambdaProb} make a choice if moving in the
Lagrangian space or in the space of the path, after that based on that
choice try to move the system in a neighbouring state - in the
Lagrangian space at
\cref{alg:annealing:changeLambda} or in the path space at
\cref{alg:annealing:changeX} - the choice is made
probabilistically in the meaning that if the energy increase in the
Lagrangian space or decrease in the path space the probability of
choosing the new state is 1, if the energy decrease in the Lagrangian
space or increase in the path space then the new state is accepted
with a probability that is\footnote{Note that $[x]^+=\max(0,x)$.}:
$$\exp(-\frac{\Delta energy}{T}).$$
Finally at the end of every trial set,
at \cref{alg:annealing:cooling}, the temperature $T$ is cooled by
a certain factor.

The $neighbour$ function choose a neighbour of the state and is
defined depending on the input
\begin{itemize}
  \item for $\lambda$ move uniformly in a range $[-maxLambdaPert, maxLambdaPert]$;
  \item for path pick randomly one of the nodes, except the extremes,
    then pick an angle in $[0,2\pi]$ and a distance
    uniformly\footnote{Or at choice with a
      Gaussian with mean in the perpendicular direction of the two
      neighbouring points.} in a specific range.
\end{itemize}

The $energy$ function is equivalent of $L_d$ in the
\cref{eq:lagrangianFunProj}:
$$energy(\mathbf{x},\lambda)=length(\mathbf{x})+\lambda\cdot
constraints(\mathbf{x})$$
or
$$energy(\mathbf{x},\lambda)=angle(\mathbf{x})+\lambda\cdot
constraints(\mathbf{x})$$
depending of the chosen method. The function $length(\mathbf{x})$
returns the total length of the path represented by the points
$\mathbf{x}$, $angle(\mathbf{x})$ is the mean complementary angle between
every pair of consecutive segments in the path $\mathbf{x}$, and
$constraints(\mathbf{x})$ return the ratio between the points inside the
obstacles and all the points of the
B-spline obtained using the path $\mathbf{x}$ as control polygon.

The annealing process
find a saddle point probabilistically increasing the energy, moving
$\lambda$ and 
decreasing the energy moving the points.


\section{Post processing}\label{sec:postPro}
The purpose of the post processing phase is to try to simplify the
path obtained in the previous phase removing useless vertexes.
\section{Degree increase}\label{sec:degreeInc}
\section{Nodes selection}\label{sec:nodeSel}
\section{Complexity}
\end{document}

%%% Local Variables:
%%% mode: latex
%%% TeX-master: "../dissertation"
%%% End:
