\documentclass{article}
\usepackage[utf8]{inputenc}
\usepackage{lmodern}
\usepackage{microtype}
\usepackage[T1]{fontenc}

\begin{document}
\section*{Slide 1}
Il mio progetto si inquadra nel path planning che consiste nel trovare
una curva, con certe caratteristiche desiderate, da un punto ad un
altro punto nello spazio, evitando intersezioni con degli
ostacoli. Nel mio lavoro mi sono concentrato nello spazio
tridimensionale, usando uno specifico tipo di curva: le curve B-spline.

Cominciamo con un paio di concetti necessari per capire i metodi
implementati.

\section*{Slide 2}
I diagrammi di Voronoi sono ricavati a partire da un insieme di punti
di input
nel piano (o nello spazio come nel mio progetto) chiamati
siti. Successivamente viene
creata una partizione dello spazio in regioni tale che ogni regione è
costituita dall'insieme di punti più vicini ad un certo sito di input
rispetto agli altri. L'insieme di queste regioni si chiama diagramma
di Voronoi. Notare che alcune celle hanno superficie infinita.

Un altro concetto importante è quello di curva B-spline.

\section*{Slide 3}
Esse sono curve parametriche polinomiali a tratti, definite in un
certo dominio parametrico $[a,b]$, per un certo grado $m$, usando una
specifica base detta base delle B-spline. Ogni curva è identificata da
un insieme di punti detti vertici di controllo, i quali formano il
poligono di controllo.

Queste curve hanno una regolarità prescritta che dipende dal grado,
seguono circa il percorso del poligono di controllo e con certi
accorgimenti possono interpolare gli estremi del poligono di controllo.

Adesso vediamo come si possono usare i diagrammi di Voronoi per
ottenere una struttura su cui identificare un percorso libero da
ostacoli.

\section*{Slide 4}
Partendo dalla scena vuota si distribuiscono uniformemente dei punti
sulle superfici degli ostacoli, e anche sulla superficie di una
bounding box che racchiude la scena. In seguito si usano tali punti
come siti di input per ricavare il diagramma di Voronoi, e si
interpreta questi come un grafo dove i nodi sono i vertici delle celle
di Voronoi e gli archi sono i lati delle celle di Voronoi pesati con
la lunghezza geometrica dei lati. I lati infiniti delle celle infinite
vengono ignorati. Infine c'è una fase di pruning dove vengono
eliminati tutti gli archi che attraversano un ostacolo. Si ottiene una
struttura che abbraccia gli ostacoli come una specie di ragnatela, e
su questa struttura è possibile attaccare con qualche regola i punti
di partenza e di arrivo desiderati e poi usare un algoritmo di routing
come quello di Dijkstra per trovare un cammino ammissibile tra questi
due punti.

Con questa operazione si ottiene un cammino che è costituito da una
poligonale spezzata.

\section*{Slide 5}
Però è più interessante avere una curva smooth. L'idea è stata quella
di usare una curva B-spline in modo tale che essa interpoli il punto
di partenza e il punto di arrivo e che usi la poligonale spezzata come
poligono di controllo. La scelta è ricaduta sulle B-spline in quanto
sono uno standard affermato nel mondo del CAD.

In questo modo però sorge un problema.

\section*{Slide 6}
Infatti il poligono di controllo è libero da ostacoli per costruzione,
infatti abbiamo fatto il pruning di tutti gli archi che intersecano
gli ostacole. Però non è detto che la curva ottenuta sia anche essa
libera da ostacoli, come si può vedere nel semplice esempio in due
dimensioni.

Per risolvere questo problema ho quindi pensato ad una soluzione che
sfrutta una proprietà delle B-spline.

\section*{Slide 7}
Una curva B-spline di grado $m$ è contenuta dentro l'unione dei convex
hull formati da $m+1$ vertici di controllo consecutivi nel poligono di
controllo. Ad esempio per una B-spline quadratica l'area in cui è
contenuta la curva è costituita dall'unione di tutti i triangoli di
vertici consecutivi sul poligono di controllo. Notare che questo vale
anche nello spazio.

Quindi per risolvere il problema della collisione della curva possiamo
usare una B-spline quadratica e far si che l'area definita da questa
regola sia libera da ostacoli. Per questo ultimo punto in particolare
ho sviluppato due soluzioni.

\section*{Slide 8}

\section*{Slide 9}

\section*{Slide 10}

\section*{Slide 11}

\section*{Slide 12}

\section*{Slide 13}

\section*{Slide 14}

\section*{Slide 15}

\section*{Slide 16}

\section*{Slide 17}

\section*{Slide 18}

\section*{Slide 19}
\end{document}

%%% Local Variables:
%%% mode: latex
%%% TeX-master: t
%%% End:
