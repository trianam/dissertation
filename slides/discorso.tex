\documentclass{article}
\usepackage[utf8]{inputenc}
\usepackage{lmodern}
\usepackage{microtype}
\usepackage[T1]{fontenc}

\newcommand{\bigO}{\ensuremath{\mathcal{O}}}

\begin{document}
\section*{Slide 1}
Il mio progetto si inquadra nel path planning che consiste nel trovare
una curva, con certe caratteristiche desiderate, da un punto ad un
altro punto nello spazio, evitando intersezioni con degli
ostacoli. Nel mio lavoro mi sono concentrato nello spazio
tridimensionale, usando uno specifico tipo di curva: le curve B-spline.

Per capire i metodi
implementati è necessario affrontare prima due argomenti.

\section*{Slide 2}
I diagrammi di Voronoi sono ricavati a partire da un insieme di punti
di input
nel piano (o nello spazio come nel mio progetto) chiamati
siti. Successivamente viene
creata una partizione dello spazio in regioni tale che ogni regione è
costituita dall'insieme di punti più vicini ad un certo sito di input
rispetto agli altri. L'insieme di queste regioni si chiama diagramma
di Voronoi. Notare che alcune celle hanno superficie infinita.

Un altro concetto importante è quello di curva B-spline.

\section*{Slide 3}
Esse sono curve parametriche polinomiali a tratti, definite in un
certo dominio parametrico $[a,b]$, per un certo grado $m$, usando una
specifica base detta base delle B-spline. Ogni curva è identificata da
un insieme di punti detti vertici di controllo, i quali formano il
poligono di controllo.

Queste curve hanno una regolarità prescritta che dipende dal grado,
seguono circa il percorso del poligono di controllo e con certi
accorgimenti possono interpolare gli estremi del poligono di controllo.

i diagrammi di Voronoi possono essere usati per
ottenere una struttura su cui identificare un percorso libero da
ostacoli.

\section*{Slide 4}
Partendo dalla scena vuota si distribuiscono uniformemente dei punti
sulle superfici degli ostacoli, e anche sulla superficie di una
bounding box che racchiude la scena. In seguito si usano tali punti
come siti di input per ricavare il diagramma di Voronoi, e si
interpreta questi come un grafo dove i nodi sono i vertici delle celle
di Voronoi e gli archi sono i lati delle celle di Voronoi pesati con
la lunghezza geometrica dei lati. I lati infiniti delle celle infinite
vengono ignorati. Infine c'è una fase di pruning dove vengono
eliminati tutti gli archi che attraversano un ostacolo. Si ottiene una
struttura che abbraccia gli ostacoli come una specie di ragnatela, e
su questa struttura è possibile attaccare con qualche regola i punti
di partenza e di arrivo desiderati e poi usare un algoritmo di routing
come quello di Dijkstra per trovare un cammino ammissibile tra questi
due punti.

Con questa operazione si ottiene un cammino che è costituito da una
poligonale spezzata.

\section*{Slide 5}
Però è più interessante avere una curva smooth. L'idea è stata quella
di usare una curva B-spline in modo tale che essa interpoli il punto
di partenza e il punto di arrivo e che usi la poligonale spezzata come
poligono di controllo. La scelta è ricaduta sulle B-spline in quanto
sono uno standard affermato nel mondo del CAD.

In questo modo però sorge un problema.

\section*{Slide 6}
Infatti il poligono di controllo è libero da ostacoli per costruzione,
perché abbiamo fatto il pruning di tutti gli archi che intersecano
gli ostacole. Però non è detto che la curva ottenuta sia anche essa
libera da ostacoli, come si può vedere nel semplice esempio in due
dimensioni.

Per risolvere questo problema ho quindi pensato ad una soluzione che
sfrutta una proprietà delle B-spline.

\section*{Slide 7}
Una curva B-spline di grado $m$ è contenuta dentro l'unione dei convex
hull formati da $m+1$ vertici di controllo consecutivi nel poligono di
controllo. Ad esempio per una B-spline quadratica l'area in cui è
contenuta la curva è costituita dall'unione di tutti i triangoli di
vertici consecutivi sul poligono di controllo. Notare che questo vale
anche nello spazio.

Quindi per risolvere il problema della collisione della curva possiamo
usare una B-spline quadratica e far si che l'area definita da questa
regola sia libera da ostacoli. Per questo ultimo punto in particolare
ho sviluppato due soluzioni.

\section*{Slide 8}
La prima soluzione consiste nell'applicare una trasformazione al grafo
ottenuto dal diagramma di Voronoi. Chiamando il grafo originale $G$ e
il grafo trasformato $G_t$, si ha che ogni tripla $(a,b,c)$ di nodi
adiacenti in
$G$ costituisce un nodo di $G_t$, e c'è un arco tra due nodi in $G_t$
se condividono un arco in $G$ ad esempio come nel caso di $(a,b,c)$ e
$(b,c,d)$. Tale arco è pesato con la distanza tra i primi due nodi in
$G$, nell'esempio $a$ e $b$. Notare che mentre $G$ non è direzionato,
$G_t$ lo è, in generale se c'è un arco tra $(a,b,c)$ e $(b,c,d)$ non
c'è un arco tra $(b,c,d)$ e $(a,b,c)$.

Dopo vi è una fase di pruning dove si eliminano
tutte i vertici di $G_t$ i cui triangoli corrispondenti intersecano un
ostacolo. Infine viene calcolato il cammino minimo con Dijkstra in
$G_t$, ricavando poi i nodi in $G$ di questo.

Guardando l'esempio schematico si ha che $G$ è costituito dai nodi
$d$, $a$, $b$, etc.. uniti come in figura. $G_t$ è costituito dalle
triple $(d,a,b)$, $(b,a,d)$, $(a,b,c)$, etc.. e nella fase di pruning
vengono eliminate le triple $(a,b,c)$ e $(c,b,a)$ perché intersecano
l'ostacolo. Eseguendo Dijkstra da una tripla speciale di inizio
collegata alle triple che contengono il nodo
iniziale a una speciale di fine collegata a quelle che contengono
il nodo finale si ottiene il cammino costituito dalle triple:
$(d,a,b)$, $(a,b,e)$, $(b,e,c)$, $(e,c,f)$; da cui si può ricavare il
cammino in $G$ costituito dai nodi: $d$, $a$, $b$, $e$, $c$, $f$.

Applicando una B-spline quadratica ad un cammino ottenuto in questo
modo abbiamo la garanzia che questa non intersechi nessun
ostacolo. Ho calcolato anche la complessità di questo metodo,

\section*{Slide 9}
che possiamo vedere riassunta in questa tabella, dove $O$ è l'insieme
degli ostacoli e $k$ è il grado massimo in $G$. Escludendo casi
limite, ci sono delle regole dei diagrammi di Voronoi che permettono
di considerare il grado massimo costante e quindi è possibile
semplificare la complessità.

Dalla tabella si evince che il costo maggiore si ha per le fasi di
pruning, dovuto al controllo delle collisioni, inoltre è possibile
dividere l'algoritmo in due parti: una fase di creazione della scena
con complessità quadratica, e poi usare la stessa scena per calcolare
curve diverse con una complessità ridotta a $\bigO(n\log n)$ che è la
stessa di Dijkstra in un grafo con grado massimo costante.

\section*{Slide 10}
Questa implementazione è interessante da un punto di vista
algoritmico, però ha il difetto di scartare diversi percorsi, con
addirittura la possibilità di rendere $G_t$ non connesso. Quindi
ho realizzato una seconda soluzione al problema in cui si calcola il
cammino con Dijkstra direttamente in $G$. In seguito c'è una fase di
controllo delle collisioni sul cammino trovato, e se una tripla del
cammino interseca un ostacolo si aggiungono vertici allineati in modo
da rendere tutti i triangoli, degenerati e no, liberi da
collisioni. Inoltre le coordinate di questi vertici allineati sono
facilmente ottenute una volta fatto il controllo di collisione. Anche
di questa soluzione ho calcolato la complessità,

\section*{Slide 11}
riassunta in questa tabella. Oltre all'insieme di ostacoli e al grado
massimo è presente l'insieme $P$ dei vertici del poligono di
controllo, che nel caso pessimo ha dimensione pari a $\bigO(|O|)$.

Anche per questo metodo è possibile considerare una fase di
costruzione della scena con complessità quadratica, e una fase di
esecuzione sulla scena con complessità maggiore rispetto alla
soluzione precedente, dovuta al termine $|P|\ |O|$ per il controllo
delle collisioni nel cammino trovato su $G$.

\section*{Slide 12}
L'uso di B-spline quadratiche ha l'inconveniente che globalmente la
curva ha continuità $C^1$, ossia è continua fino alla derivata
prima. In alcune applicazioni questo non è
sufficiente, se ad esempio vogliamo usare la curva come supporto per il
moto di un oggetto vogliamo continuità almeno fino alla derivata
seconda.

Se manteniamo il poligono di controllo invariato e aumentiamo il grado
della curva si ha il problema che la proprietà di convex hull non vale
più come prima, l'area in cui si può trovare la curva non è più unione
di triangoli ma unione di tetraedri o comunque figure solide. In
questo modo non possiamo più fare il controllo di collisione con i
metodi descritti prima. La soluzione che ho trovato consiste
nell'aggiungere vertici allineati nel poligono di controllo in modo da
non dover modificare i metodi visti prima. Per descrivere il modo in
cui vengono aggiunti questi nuovi vertici consideriamo un esempio.

\section*{Slide 13}
Il poligono di controllo iniziale è costituito dai vertici in verde, e
l'area in cui si può trovare una B-spline quadratica è evidenziata in
celeste. Se vogliamo aumentare il grado a 4 è necessario aggiungere
due vertici per segmento. In questo modo e possibile aumentare il
grado perché l'area in cui si può trovare una curva quartica è
costituita dall'unione di convex hull di 5 vertici consecutivi, e in
questo caso sono tutti triangoli contenuti nell'area precedente. Se
precedentemente avevamo usato uno dei due metodi visti per garantire la non
collisione della curva con gli ostacoli, a maggior ragione la curva
quartica così ottenuta sarà libera da collisioni.

Oltre all'incremento di grado, ho previsto anche una fase opzionale di
post processing.

\section*{Slide 14}
Con lo scopo di semplificare il poligono di controllo eliminando i
vertici inutili, e quindi evitando curve inutili nella B-spline.

Questa fase consiste nel considerare tutte le triple di vertici
consecutivi nel poligono di controllo, e, se il triangolo
corrispondente non interseca nessun ostacolo, allora si semplifica la
tripla eliminando il vertice centrale. Prima di fare questa
semplificazione però, è necessario controllare che la modifica non
influisca negativamente sulle triple adiacenti.

Infine, oltre agli algoritmi visti, ho implementato anche una
soluzione che consiste in un metodo di ottimizzazione stocastico.

\section*{Slide 15}
Il problema del path planning può essere visto come un problema di
ottimizzazione vincolata, dove si deve minimizzare una grandezza di
interesse al variare della
posizione dei vertici di controllo, con il vincolo che la curva
ottenuta sia libera da collisioni con gli ostacoli. La grandezza di
interesse è una 
combinazione di lunghezza della curva, picco di curvatura e picco di
torsione.

Per risolvere questo tipo di problemi si può usare la cosiddetta
lagrangian relaxation, che consiste nel rilassare il vincolo
aumentando la dimensione dello spazio degli stati, e creando una
superficie così definita dove gain è la grandezza di interesse e
constraint è di quanto la curva interseca gli ostacoli. Su tale
superfice si trova il minimo
desiderato tra i punti di sella usando un metodo di ottimizzazione
come il simulated annealing.

I drawback di questo metodo sono la lentezza di esecuzione e il fatto
che sia il calcolo della quantità da minimizzare che del vincolo sono
effettuati in modo discreto, quindi lasciando spazio ad errori in casi
particolari.

Per sviluppare gli algoritmi visti ho realizzato una applicazione
object oriented.

\section*{Slide 16}
In python 3, usando la libreria SciPy per i calcoli vettoriali, di algebra
lineare e il calcolo delle curve B-spline; la libreria NetworkX per la
rappresentazione di grafi e le operazioni su essi; e VTK per
l'interfaccia grafica.

E ho usato l'applicazione per eseguire diversi test di cui evidenzio

\section*{Slide 17}
Una esecuzione del primo metodo senza post processing. Una esecuzione
del secondo metodo sempre senza post processing in cui si nota che la
curva segue un percorso che il primo metodo aveva scartato. Una
esecuzione del primo metodo questa volta con il post processing di cui
si può apprezzare l'efficacia in termini di semplificazione della
curva. Una esecuzione del secondo metodo con il post processing. E
infine una esecuzione del metodo di ottimizzazione stocastico dove si
può apprezzare la minore lunghezza della curva.

Alcuni possibili sviluppi futuri possono essere:

\section*{Slide 18}
Cambiare la struttura di base, ad esempio cambiando il modo in cui si
attaccano i punti di partenza e arrivo per evitare ganci indesiderati,
oppure usare un altro metodo invece di quello basato sui diagrammi di
Voronoi, ad esempio usando il visibility graph o gli RRT o altri.

Un altro miglioramento può riguardare l'aumento di grado, l'uso di
vertici allineati infatti ha lo svantaggio di forzare la curva a fare
dei salti di torsione, una soluzione può essere quella ad esempio di
considerare una soluzione tipo il metodo 2 dove però si controllano
non più i triangoli nel poligono di controllo, ma i poliedri di 4 o 5
vertici consecutivi.

È possibile anche migliorare la fase di post processing in quanto
l'attuale ha lo svantaggio di non essere simmetrica. Se si trova una
curva da un punto $a$ ad un punto $b$ in generale sarà diversa dalla
curva trovata dal punto $b$ al punto $a$.

È possibile anche concentrarsi su altri metodi di ottimizzazione, ad
esempio usando come stato iniziale del sistema una soluzione
ammissibile ottenuta con uno dei due 
metodi proposti, e facendo evolvere il sistema solo in stati
ammissibili.

\section*{Slide 19}
Con questo ho concluso, se ci sono domande, grazie.

\end{document}

%%% Local Variables:
%%% mode: latex
%%% TeX-master: t
%%% End:
