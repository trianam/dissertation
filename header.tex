\documentclass[twoside,openright,titlepage,fleqn,
	headinclude,12pt,a4paper,BCOR5mm,footinclude
	]{scrbook}
%\documentclass[a4paper,11pt]{book}
\usepackage[utf8]{inputenc}
%\usepackage[latin1]{inputenc}
\usepackage{lmodern}
\usepackage{microtype}
\usepackage[T1]{fontenc}
\usepackage{makeidx}
%\usepackage[italian]{babel}
%\usepackage{graphicx}
%\usepackage{hyperref}
\usepackage{lscape}
\usepackage{siunitx}
\usepackage{enumitem}
\usepackage{subfiles}
\usepackage{listings}
\usepackage{algorithm}
\usepackage{algpseudocode}
%\usepackage{pgf}
\usepackage{xparse}
\usepackage{float}
\usepackage{color}
%\usepackage{xcolor}

%pacchetti del modello per la tesi
%--------------------------------------------------------------
\usepackage[square,numbers]{natbib} 
\usepackage[fleqn]{amsmath}  
\usepackage{dia-classicthesis-ldpkg} 
\usepackage{mathtools}
\usepackage{amsfonts}
\usepackage{amssymb}
\usepackage{amsthm}
\usepackage{stmaryrd}
\usepackage{empheq}
\usepackage{pifont}
% Options for classicthesis.sty:
% tocaligned eulerchapternumbers drafting linedheaders 
% listsseparated subfig nochapters beramono eulermath parts 
% minionpro pdfspacing
\usepackage[linedheaders,eulerchapternumbers,subfig,beramono,eulermath,parts]{classicthesis}
%--------------------------------------------------------------
\usepackage{tikz}
\usetikzlibrary{shapes, chains, scopes, shadows, positioning, arrows,
  decorations.pathmorphing, calc, mindmap, petri}

\usepackage[capitalise]{cleveref}
\usepackage{acronym}

% make the part links in the TOC point to the right page
\usepackage{etoolbox}
\pretocmd{\oldpart}{\cleardoublepage}{}{}

\makeatletter
\def\ttl@tocpart{% %magic code, don't ask
  \def\ttl@a{\protect\numberline{\thepart}\@gobble{}}}

%% If you want part titles aligned with chapter titles
%% uncomment the following line and remove the code
%% up to (and excluding) \makeatother
 \setlength{\cftpartnumwidth}{\cftchapnumwidth}

% \let\classic@l@part\l@part
% \renewcommand\l@part[2]{%
%   \begingroup
%   \renewcommand{\numberline}[1]{\textsc{##1} }%
%   \classic@l@part{#1}{#2}%
%   \endgroup
% }
\makeatother

% %%% Magic code for part indices; don't ask ;-)
% \makeatletter
% \def\ttl@tocpart{%
%   \def\ttl@a{\protect\numberline{\thepart}\@gobble{}}}
% \makeatother

%comandi per modello
%--------------------------------------------------------------
\newcommand{\myTitle}{B-Spline methods for the design of smooth spatial
  paths with obstacle avoidance\xspace}
\newcommand{\mySubtitle}{Metodi B-spline per il disegno di percorsi
  regolari in ambienti tridimensionali contenenti ostacoli}
\newcommand{\mySubsubtitle}{Tesi di Laurea Magistrale in Informatica}
% use the right myDegree option
\newcommand{\myDegree}{Corso di Laurea Magistrale in Informatica \xspace}
\newcommand{\myName}{Stefano Martina\xspace}
\newcommand{\myMail}{stefano.martina@stud.unifi.it}
\newcommand{\myProf}{Alessandra Sestini\xspace}
%\newcommand{\myOtherProf}{Carlotta Giannelli\xspace}
\newcommand{\mySupervisor}{Carlotta Giannelli\xspace}
\newcommand{\myFaculty}{Scuola di Scienze Matematiche Fisiche e Naturali\xspace}
\newcommand{\myDepartment}{
	Department of mathematics and computer science\xspace}
\newcommand{\myUni}{\protect{
	Universit\`a degli Studi di Firenze}\xspace}
\newcommand{\myLocation}{Firenze\xspace}
\newcommand{\myAcademicYear}{2015/16\xspace}
\newcommand{\myMonth}{Luglio\xspace}
\newcommand{\myYear}{2016\xspace}
\newcommand{\myVersion}{Version 0.1\xspace}
\newcommand{\mycopyright}{
  this work is licensed under a
  \href{http://creativecommons.org/licenses/by-sa/4.0/}{Creative
    Commons Attribution-ShareAlike 4.0 International License \includegraphics[width=1cm]{logo/logoCC.png}}\xspace}
\newlength{\abcd} % for ab..z string length calculation
% how all the floats will be aligned
\newcommand{\myfloatalign}{\centering} 
\setlength{\extrarowheight}{3pt} % increase table row height
\captionsetup{format=hang,font=small}
%--------------------------------------------------------------
% Layout setting
%--------------------------------------------------------------
\usepackage{geometry}
\geometry{
	a4paper,
	ignoremp,
	bindingoffset = 1cm, 
	textwidth     = 13.5cm,
	textheight    = 21.5cm,
	lmargin       = 3.5cm, % left margin
	tmargin       = 4cm    % top margin 
}
%--------------------------------------------------------------

%comando per impostazioni float di default
\makeatletter
\def\fps@figure{!htbp}
\def\fps@table{!htbp}
\def\fps@code{!htbp}
\makeatother

%cambia comportamento delle description
\renewcommand{\descriptionlabel}[1]{\hspace{2em}\hspace{\labelsep}\textbf{#1}}


%definisce un nuovo ambiente float per il codice
\newfloat{code}{!htbp}{}
\floatname{code}{Codice}

%comando per dimensioni testo
\newcommand{\dimg}{\tiny}
\newcommand{\codg}[1]{\dimg \unicocodet{#1}}
%  \lstinline[basicstyle=\dimg\ttfamily\bfseries,breaklines=true]|#1|


\newcommand{\dims}{\scriptsize}
\newcommand{\cods}[1]{\dims\unicocodet{#1}}
%  \lstinline[basicstyle=\dims\ttfamily\bfseries,breaklines=true]|#1|

%comando per inserire un link
\newcommand{\link}[1]{\unicocode{#1}}


%comando per inserire un file
\newcommand{\file}[1]{\unicocode{#1}}

\newcommand{\bigO}{\ensuremath{\mathcal{O}}}

\def\transW{8mm}
\def\transH{2mm}

\tikzstyle{figureFrame}=[rectangle, rounded corners=2pt, inner sep = 0.3cm, drop shadow, draw=black!50, fill=white, anchor=south west, inner sep=0]
\tikzstyle{imageLabel}=[text=red]
\tikzstyle{imageArrow}=[draw=red, line width=0.5mm, ->]
\tikzstyle{obstacle}=[fill=yellow!50, draw=black]
\tikzstyle{convexHull}=[fill=blue!30]
\tikzstyle{convexHullBord}=[color=black, dotted]%dash pattern=on 3pt off 3pt]
\tikzstyle{knot}=[color=red, draw=black]
\tikzstyle{knotPoly}=[color=black, line width=0.25mm]
\tikzstyle{controlToKnot}=[color=black, dotted]
\tikzstyle{controlPoly}=[color=black, line width=0.25mm]
\tikzstyle{controlPolyHigh}=[color=red, line width=0.25mm]
\tikzstyle{controlPolyTract}=[color=black, line width=0.25mm, dash pattern=on 3pt off 3pt]
\tikzstyle{controlPolyTractHigh}=[color=red, line width=0.25mm, dash pattern=on 3pt off 3pt]
\tikzstyle{obstacleTract}=[draw=black, line width=0.25mm, dash pattern=on 3pt off 3pt]
\tikzstyle{spline}=[color=red, line width=0.5mm]
\tikzstyle{controlVert}=[color=green, draw=black]
\tikzstyle{controlVertHigh}=[color=red, draw=black]
\tikzstyle{obstaclePoint}=[color=red, draw=black]
\tikzstyle{site}=[color=blue, draw=black]
\tikzstyle{siteHigh}=[color=red, draw=black]
\tikzstyle{textArrow}=[draw=red, line width=0.5mm, ->]
\tikzstyle{vertex}=[color=green, draw=black]
\tikzstyle{vertexHigh}=[color=red, draw=black]
\tikzstyle{intersection}=[color=red, draw=black]
\tikzstyle{poly}=[color=black, line width=0.25mm]
\tikzstyle{polyTract}=[color=black, line width=0.25mm, dash pattern=on 3pt off 3pt]
\tikzstyle{cutting}=[color=red, line width=0.25mm]

\colorlet{mmcb}{black!70}
\colorlet{mmc1}{red!80}
\colorlet{mmc2}{blue!80}
\colorlet{c1}{green!20}
\colorlet{c2}{blue!10}
\colorlet{c3}{yellow!10}
\colorlet{c4}{red!10}
\colorlet{drawColor}{black!80}
\colorlet{commentColor}{green!70!black!90}
\colorlet{codeBgColor}{yellow!50}
\colorlet{bashBgColor}{green!50}

\tikzset{onslide/.code args={<#1>#2}{%
  \only<#1>{\pgfkeysalso{#2}} % \pgfkeysalso doesn't change the path
}}
\tikzset{temporal/.code args={<#1>#2#3#4}{%
  \temporal<#1>{\pgfkeysalso{#2}}{\pgfkeysalso{#3}}{\pgfkeysalso{#4}} % \pgfkeysalso doesn't change the path
}}

\tikzstyle{alertStar}=[circle, decorate, decoration={zigzag,segment length=3.12mm,amplitude=1mm}, align=center, drop shadow, draw=drawColor, fill=white]
\tikzstyle{oval}=[ellipse, align=center, drop shadow, draw=drawColor, fill=white]
\tikzstyle{rect}=[rectangle, rounded corners=2pt, align=center, drop
shadow, draw=drawColor, fill=white]
\tikzstyle{arrow}=[->, very thick, >=stealth', draw=black!80]
\tikzstyle{myMindmap}=[mindmap,
every node/.style={concept, minimum size=5mm, text width=5mm}, 
% every child/.style={level distance=10mm, concept color=mmcb}
level 1/.append style={level distance=10mm,sibling angle=45},
level 2/.append style={level distance=10mm,sibling angle=45},
level 3/.append style={level distance=10mm,sibling angle=45}
]
\tikzstyle{myPlace} = [place, very thick, draw=drawColor, fill=white, drop shadow]
\tikzstyle{transExpH} = [transition, very thick, draw=drawColor, fill=white, drop
shadow, minimum width=\transW, minimum height=\transH]
\tikzstyle{transExpV} = [transition, very thick, draw=drawColor, fill=white, drop
shadow, minimum width=\transH, minimum height=\transW]
\tikzstyle{transDetH} = [transition, very thick, draw=drawColor, fill=black, drop shadow, minimum width=\transW, minimum height=\transH]
\tikzstyle{transDetV} = [transition, very thick, draw=drawColor, fill=black, drop shadow, minimum width=\transH, minimum height=\transW]
\tikzstyle{pre}=[<-, very thick, >=stealth', draw=drawColor]
\tikzstyle{preN}=[<-, very thick, >=o, draw=drawColor]
\tikzstyle{post}=[->, very thick, >=stealth', draw=drawColor]
\tikzstyle{highlight}=[draw=red]

\NewDocumentEnvironment{myfig}{mm}{
  \begin{figure}
    \begin{center}
    }{
    \end{center}
    \caption{#1}
    \label{#2}
  \end{figure}
}

%comando per inserire una immagine
\NewDocumentCommand{\image}{mmmo}{
  \begin{figure}
    \begin{center}
      \begin{tikzpicture}
        \node [figureFrame] (image) at (0,0) {\includegraphics[width=.9\textwidth]{#1}};
        \IfNoValueTF{#4}
        {}
        {
          \begin{scope}[x={(image.south east)},y={(image.north west)}]
            #4
          \end{scope}
        }
      \end{tikzpicture}
    \end{center}
    \caption{#2}
    \label{#3}
  \end{figure}
}

\NewDocumentCommand{\imagev}{mmmo}{
  \begin{figure}
    \begin{center}
      \begin{tikzpicture}
        \node [figureFrame, inner sep=5mm] (image) at (0,0) {\includegraphics[height=.9\textheight]{#1}};
        \IfNoValueTF{#4}
        {}
        {
          \begin{scope}[x={(image.south east)},y={(image.north west)}]
            #4
          \end{scope}
        }
      \end{tikzpicture}
    \end{center}
    \caption{#2}
    \label{#3}
  \end{figure}
}

\NewDocumentCommand{\imaget}{mmmmmo}{
  \begin{figure}
    \begin{center}
      \begin{tikzpicture}
        \node [figureFrame] (image1) at (0,0) {\includegraphics[width=.9\textwidth]{#1}};
        \node [figureFrame, below=of image1] (image2) {\includegraphics[width=.9\textwidth]{#2}};
        \node [figureFrame, below=of image2] (image3) {\includegraphics[width=.9\textwidth]{#3}};
        \IfNoValueTF{#6}
        {}
        {
          \begin{scope}[x={(image.south east)},y={(image.north west)}]
            #6
          \end{scope}
        }
      \end{tikzpicture}
    \end{center}
    \caption{#4}
    \label{#5}
  \end{figure}
}

\NewDocumentCommand{\imagett}{mmmmmmo}{
  \begin{figure}
    \begin{center}
      \begin{tikzpicture}
        \node [figureFrame] (image1) at (0,0) {\includegraphics[width=.9\textwidth]{#1}};
        \node [figureFrame, below=of image1] (image2) {\includegraphics[width=.9\textwidth]{#2}};
        \coordinate[below=2.5cm of image2] (puntello);
        \node [figureFrame, left = 0.2cm of puntello] (image3) {\includegraphics[width=.45\textwidth]{#3}};
        \node [figureFrame, right = 0.2cm of puntello] (image4) {\includegraphics[width=.45\textwidth]{#4}};
        \IfNoValueTF{#7}
        {}
        {
          \begin{scope}[x={(image.south east)},y={(image.north west)}]
            #7
          \end{scope}
        }
      \end{tikzpicture}
    \end{center}
    \caption{#5}
    \label{#6}
  \end{figure}
}

\NewDocumentCommand{\imageDouble}{mmmmoo}{
  \begin{figure}
    \begin{center}
      \begin{tikzpicture}
        \node [figureFrame] (imageL) at (0,0) {\includegraphics[width=.45\textwidth]{#1}};
        \IfNoValueTF{#5}
        {}
        {
          \begin{scope}[x={(imageL.south east)},y={(imageL.north west)}]
            #5
          \end{scope}
        }
      \end{tikzpicture}\rule{5mm}{0mm}
      \begin{tikzpicture}
        \node [figureFrame] (imageR) at (0,0) {\includegraphics[width=.45\textwidth]{#2}};
        \IfNoValueTF{#6}
        {}
        {
          \begin{scope}[x={(imageR.south east)},y={(imageR.north west)}]
            #6
          \end{scope}
        }
      \end{tikzpicture}
    \end{center}
    \caption{#3}
    \label{#4}
  \end{figure}
}

\NewDocumentCommand{\imager}{mmmo}{
  \begin{figure}
    \begin{center}
      \begin{tikzpicture}
        \node [figureFrame, rotate = 90] (image) at (0,0) {\includegraphics[height=.9\textwidth]{#1}};
        \IfNoValueTF{#4}
        {}
        {
          \begin{scope}[x={(image.south east)},y={(image.north west)}]
            #4
          \end{scope}
        }
      \end{tikzpicture}
    \end{center}
    \caption{#2}
    \label{#3}
  \end{figure}
}

\lstdefinestyle{customPython}{
   language=Python,
   % basicstyle=\small\ttfamily\bfseries,
   basicstyle=\tiny\ttfamily,
   keywordstyle=\color{blue}\ttfamily,
   stringstyle=\color{red}\ttfamily,
   commentstyle=\color{green}\ttfamily,
   morecomment=[l][\color{magenta}]{\#},
   % breaklines=false,
   breaklines=true, breakatwhitespace=true,
   postbreak=\raisebox{0ex}[0ex][0ex]{\ensuremath{\color{red}\hookrightarrow\space}},
   frameround=fttt,
   frame=trBL,
   backgroundcolor=\color{yellow!20},
   numbers=left,
   stepnumber=1,    
   firstnumber=1,
   numberfirstline=true,
   numberstyle=\tiny\color{black!50},
   xleftmargin=1.75em,
   framexleftmargin=2.1em,
   % rulesepcolor=\color{gray},
   rulecolor=\color{black}
   % linewidth=8cm,
}

\lstdefinestyle{customInlinePython}{
   language=Python,
   % basicstyle=\small\ttfamily\bfseries,
   basicstyle=\ttfamily,
   keywordstyle=\color{blue}\ttfamily,
   stringstyle=\color{red}\ttfamily,
   commentstyle=\color{green}\ttfamily,
   morecomment=[l][\color{magenta}]{\#}
}

\lstnewenvironment{pblock}[1][]
{
  \lstset{
    style=customPython,
    #1
  }
}{}

\newcommand{\pfile}[2][]{
  \lstinputlisting[style=customPython, title={\texttt{\detokenize{#2}}}, #1]{#2}
}

\newcommand{\pp}[2][]{\lstinline[style=customInlinePython,#1]`#2`}
  %\colorbox{codeBgColor}{
  %  \lstinline[style=customPython,#1]`#2`
  %}
%}

\graphicspath{{img/}}
\lstset{inputpath=src/}

%% \definecolor{links}{HTML}{2A1B81}
%% \hypersetup{colorlinks,linkcolor=links,urlcolor=links}

%% \definecolor{links}{HTML}{2A1B81}
%% \hypersetup{colorlinks,linkcolor=,urlcolor=links}

\newcommand{\me}{\ensuremath{\mathrm{e}}}
\newcommand{\md}{\ensuremath{\mathrm{d}}}
\newcommand{\tc}{\ensuremath{\mathrm{t.c.:}\quad}}
\newcommand{\expected}[1]{\ensuremath{\mathrm{\textbf{E}}\left[#1\right]}}
\newcommand{\variance}[1]{\ensuremath{\mathrm{\textbf{Var}}\left(#1\right)}}
\newcommand{\prob}[1]{\ensuremath{\mathrm{\textbf{P}}\left(#1\right)}}
%\newcommand{\max}[1]{\ensuremath{\mathrm{max}\left(#1\right)}}
\newcommand{\abs}[1]{\ensuremath{\left|#1\right|}}
\newcommand{\mE}{\ensuremath{\mathbb{E}}}
\newcommand{\mR}{\ensuremath{\mathbb{R}}}
\newcommand{\mN}{\ensuremath{\mathbb{N}}}
\newcommand{\mS}{\ensuremath{\mathbb{S}}}
\newcommand{\conv}{\ensuremath{\mathbf{Conv}}}
\newcommand{\ve}[1]{\ensuremath{\boldsymbol{#1}}}
\newcommand{\overmath}[2]{\ensuremath{\stackrel{#1}{#2}}}
\newcommand{\norm}[1]{\left\lVert#1\right\rVert_2}

\newcommand{\bs}{B-spline\xspace}
\newcommand{\bss}{B-splines\xspace}

\newcommand{\cmark}{\ding{51}}
\newcommand{\xmark}{\ding{55}}

\newcommand{\degTwo}{2\xspace}
\newcommand{\degThree}{3\xspace}
\newcommand{\degFour}{4\xspace}
\newcommand{\metA}{A\xspace}
\newcommand{\metB}{B\xspace}
\newcommand{\metC}{C\xspace}
%\newcommand{\ly}{\xspace}
\newcommand{\ypp}{\cmark\xspace}
\newcommand{\npp}{\xmark\xspace}
\newcommand{\akp}{\textbf{A}\xspace}
\newcommand{\ukp}{\textbf{U}\xspace}
\newcommand{\nd}{\textbf{-}\xspace}
\newcommand{\testLab}[7]{Test #1; Scene #2; $\ve{s}\shortrightarrow\ve{e}$ #3; Deg. #4; Meth. #5; Post proc. #6; Part. #7.}
\newcommand{\testLabAnn}[7]{Test #1; Scene #2; $\ve{s}\shortrightarrow\ve{e}$ #3; Deg. #4; Meth. #5; Part. #6; Config. #7.}
\newcommand{\lenArc}{arc}
\newcommand{\lenPol}{poly}
\newcommand{\ratios}[3]{$[#1, #2, #3]$}
\newcommand{\vertex}[3]{$[#1, #2, #3]$}
\newcommand{\vertices}[6]{$\scriptstyle[#1, #2, #3]\shortrightarrow[#4, #5, #6]$}
\newcommand{\sceneA}{1\xspace}
\newcommand{\sceneAb}{1b\xspace}
\newcommand{\sceneB}{2\xspace}
\newcommand{\sceneBb}{2b\xspace}
\newcommand{\sceneC}{3\xspace}
\newcommand{\annA}{1\xspace}
\newcommand{\annB}{2\xspace}
\newcommand{\annC}{3\xspace}



\makeatletter
\newcommand{\pushright}[1]{\ifmeasuring@#1\else\omit\hfill$\displaystyle#1$\fi\ignorespaces}
\newcommand{\pushleft}[1]{\ifmeasuring@#1\else\omit$\displaystyle#1$\hfill\fi\ignorespaces}
\makeatother


\crefname{problem}{problem}{problems}
\crefname{void}{}{}
\crefname{system}{system}{systems}
\crefformat{part}{Part #2\MakeUppercase{#1}#3}

%comando per inserire algoritmo
\NewDocumentEnvironment{algo}{mm}{
\begin{algorithm}
  \caption{#1}\label{#2}
  \begin{algorithmic}[1]
}{
 \end{algorithmic}
\end{algorithm}
}

%algorithmicx keywords
\algnewcommand\Ass{\ensuremath{\gets}}
\algnewcommand\True{\ensuremath{True}}
\algnewcommand\False{\ensuremath{False}}
\algnewcommand\Break{\textbf{break}}
\algnewcommand\IfNot{\textbf{not}\xspace}
\algnewcommand\IfAnd{\textbf{and}\xspace}
\algnewcommand\IfOr{\textbf{or}\xspace}

\sisetup{output-exponent-marker=\ensuremath{\mathrm{e}}}

\makeindex

%%% Local Variables:
%%% mode: latex
%%% TeX-master: "dissertation"
%%% End:
